% !TeX root = document.tex
Cilj ovog kratkog rada je upoznati čitatelje s osnovama i zainteresirati za daljnje proučavanje naglasnog sustava standardnog tokijskog dijalekta japanskog jezika.
Japanski naglasni sustav je dosta različit od hrvatskog ili engleskog, stoga je bitno najprije upoznati se s temeljima tog sustava definiranim u ovom odjeljku.
U idućem odjeljku dan je sustavni pregled tipova naglasaka kroz primjere u kojima se pojavljuju, uz poneko pravilo i najčešće iznimke koje se pojavljuju u svakodnevnom govoru.
Ostali odjeljci obrađuju pravila vezana uz naglaske kod imenica, pridjeva i glagola.

Pa, što bi Japanci rekli, よろしくおねがいします!

\subsection{M\=ora}
M\=ora je japanski ekvivalent sloga.
To je ujedno i najmanja nedjeljiva izgovorna jedinica neke riječi.
Karakteristično obilježje japanskog jezika jest da svaka m\=ora ima jednako trajanje.
U načelu, jedna m\=ora odgovara jednoj hiragani, osim malih ゃ, ゅ, ょ kada su nalijepljeni na drugu hiraganu, npr. u しゃ, しゅ, しょ.
Dakle, \textit{samoglasno N} ん se broji kao 1 m\=ora.
I \textit{sokuon} っ (npr. u げっかん) i \textit{ch\=oonpo} ー  (npr. u ハート) se broje kao 1 m\=ora.
U tablici \ref{tab:mora-type} ispod kopiranoj s Wikipedije navedene su vrste m\=ora za radoznale:

\begin{table}[htb]
	\centering
	\caption{Vrste m\=ora.}
	\label{tab:mora-type}
	\begin{tabular}{clllc}
		\toprule
		\multicolumn{2}{l}{Vrsta} & Primjer & & \makecell[b]{Broj m\=ora \\ u primjeru} \\
		\midrule
		V & samoglasnici & \e{お} & & 1 \\
		jV & palatalizirani samoglasnici & \e{よ} &(世) & 1 \\
		CV & suglasnik i samoglasnik & \e{こ} &(子) & 1 \\
		CjV & suglasnik i palatalizirani samoglasnik & \e{きょ} &(巨) & 1 \\
		R & produljenje samoglasnika & きょ\e{う} &(今日)& 2 \\
		& & キ\e{ー} & & 2 \\
		N & samoglasno N & こ\e{ん} &(紺) & 2 \\
		Q & udvostručavanje suglasnika (geminacija) & こ\e{っ}こ &(国庫) & 3 \\
		\bottomrule
	\end{tabular}
\end{table}

Određena riječ može imati naglasak na najviše jednoj m\=ori.
Od svih vrsta, samo prve četiri (V, jV, CV i CjV) mogu nositi naglasak.

\subsection{Tonovi, intonacija i naglasak}
U japanskom tehnički razlikujemo samo dvije vrste tona: visoki i niski.
Svaka m\=ora ima točno jedan od ta dva tona, ton se ne može mijenjati unutar iste m\=ore.
Pri promjeni iz niskog tona u visoki ili obratno se osjeti razlika u visini; međutim, i među tonovima iste vrste visina također može blago varirati.
Prirodno u izgovoru visina opada prema kraju naglasne cjeline.

Intonacija je kretanje visine tonova u izgovorenoj riječi.
Intonacija za sve riječi koje nemaju naglasak je ista --- neutralna, a za riječi koje naglasak imaju ovisi o mjestu gdje se naglasak nalazi.
U japanskom jeziku neutralna intonacija je pomalo neobična: počinje s niskim tonom na prvoj m\=ori, a sve m\=ore poslije prve imaju visoki ton.
Primjeri:
\begin{multicols}{3}
	\begin{itemize}
		\item \pitch{に/0,ほ/1,ん/1,ご/1}(日本語)
		\item \pitch{こ/0,ん/1,に/1,ち/1,は/1}
		\item \pitch{く/0,る/1,ま/1}(車)
		\item \pitch{り/0,ん/1,ご/1}
		\item \pitch{し/0,ん/1,ぶ/1,ん/1}(新聞)
		\item \pitch{と/0,も/1,だ/1,ち/1}(友達)
	\end{itemize}
\end{multicols}

U japanskom jeziku naglasak se manifestira tako što nakon m\=ore koja je naglašena visina intonacije naglo padne, odnosno naglašena m\=ora je posljednja visoka u riječi.
Primjeri (naglašena m\=ora je podebljana):
\begin{multicols}{3}
	\begin{itemize}
		\item \pitch{\e{ね}/1,こ/0}(猫)
		\item \pitch{な/0,\e{が}/1,さ/0,き/0}(長崎)
		\item \pitch{に/0,ほ/1,ん/1,\e{じ}/1,ん/0}(日本人)
		\item \pitch{い/0,も/1,う/1,\e{と}/1}(妹)
		\item \pitch{コ/0,ー/1,\e{ヒ}/1,ー/0}
		\item \pitch{ピ/0,\e{カ}/1,チ\m{ュ}/0,ウ/0}
	\end{itemize}
\end{multicols}

\subsection{Naglasna cjelina}
Naglasnu cjelinu čini niz riječi koji zajedno ima najviše jedan naglasak.
Svaka riječ inherentno ima ili nema određeni naglasak, no kad se više riječi udruži u naglasnu cjelinu, naglasak u pojedinim riječima može se izgubiti ili pomaknuti prema određenim pravilima.
Najčešće se radi o glavnoj riječi sa zanaglasnicama (čestice, kopula, pomoćni glagoli), ili o konjugiranom obliku neke riječi.
Neki primjeri naglasnih cjelina (naglašena m\=ora, gdje postoji, je podebljana):
\begin{itemize}
	\begin{multicols}{2}
	\item \pitch{こ/0,れ/1}+\pitch{は/0}=\pitch{こ/0,れ/1,は/1}
	\item \pitch{た/0,な/1,か/1}+\pitch{\e{で}/1,す/0}=\pbox{\textwidth}{\pitch{た/0,な/1,か/1,\e{で}/1,す/0}\bh(田中です)}
	\item \pitch{お/0,と/1,\e{こ}/1}+\pitch{\e{で}/1,す/0}=\pbox{\textwidth}{\pitch{お/0,と/1,\e{こ}/1,で/0,す/0}\bh(男です)}
	\item \pitch{\e{じ}/1,し\m{ょ}/0}+\pitch{\e{で}/1,す/0}=\pbox{\textwidth}{\pitch{\e{じ}/1,し\m{ょ}/0,で/0,す/0}\bh(辞書です)}
	\item \pbox{\textwidth}{\pitch{あ/0,た/1,た/1,\e{か}/1,い/0}\bh(暖かい)}+\pitch{て/0}=\pbox{\textwidth}{\pitch{あ/0,た/1,\e{た}/1,か/0,く/0,て/0}\bh(暖かくて)}\footnote{više o naglasku kod idjeva u pododjeljku \ref{sub:idjevi}}
	\end{multicols}
	\item \pitch{た/0,\e{べ}/1,る/0}+\pitch{さ/0,せ/1,る/1}+\pitch{ら/0,れ/1,る/1}
=\pbox{\textwidth}{\pitch{た/0,べ/1,さ/1,せ/1,ら/1,\e{れ}/1,る/0}\bh(食べさせられる)}
\end{itemize}

%	\begin{center}
%	\begin{tabular}{cccccc}
%		\pitch{こ/0,れ/1} & + & \pitch{は/0} & = & \pitch{こ/0,れ/1,は/1} & \\
%		\pitch{た/0,な/1,か/1} & + & \pitch{\e{で}/1,す/0} & = & \pitch{た/0,な/1,か/1,\e{で}/1,す/0} &(田中です)\\
%		\pitch{お/0,と/1,\e{こ}/1} & + & \pitch{\e{で}/1,す/0} & = & \pitch{お/0,と/1,\e{こ}/1,で/0,す/0} &(男です)\\
%	\end{tabular}
%	\end{center}
