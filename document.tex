\documentclass[12pt]{extarticle}
\usepackage[CJKspace=true]{xeCJK}
\usepackage{fontspec}

\usepackage{booktabs}
\usepackage{tabularx}
\usepackage{ruby}
\usepackage{tipa}
\usepackage{tipx}
\usepackage{tikz}
\usepackage{amsmath}

\usepackage[croatian]{babel}
\usepackage{indentfirst}
\usepackage[a4paper, margin=2.5cm]{geometry}
\usepackage{enumitem}
\usepackage[all]{nowidow}
\usepackage{setspace}
\usepackage{microtype}
\usepackage{multicol}
\usepackage{makecell}

\setlist[itemize]{itemsep=0em, topsep=0.5em}
\setlist[enumerate]{itemsep=0em, topsep=0.5em}

\setCJKmainfont{Noto Serif CJK JP}
\setCJKsansfont{Noto Sans CJK JP}
\setmainfont[Numbers={Lining, Monospaced}, Scale=1.1]{Brill}
\renewcommand{\rubysep}{-0.5ex}
\renewcommand{\rubysize}{0.5}
\renewcommand{\arraystretch}{1.25}

\newcommand{\pitch}[1]{\raisebox{-\dp\strutbox}{
	\begin{tikzpicture}[
		every node/.style={inner sep=0.0, outer sep=0.0},
		dot/.style={draw, fill, circle, minimum size=0.5ex}
	]
	\foreach[count=\i from 1] \x/\p in {#1} 
		\draw node (P\i) at (\i * 1.1em, 0) {\x\strut} node[dot, anchor=base] (C\i) at (\i * 1.1em, \p * 0.4em + 0.8em) {};
	
	\foreach[count=\i from 1, evaluate=\i as \j using int(\i - 1)] \x/\p in {#1}
		\ifnum \j>0
			\draw ({C\i}.center) -- ({C\j}.center)
		\fi;
	\end{tikzpicture}}
}

\newcommand{\e}{\textbf}
\newcommand{\m}[1]{\kern -0.3em{#1}}

\title{Uvod u japanski naglasni sustav}
\author{Lovriša}

\begin{document}
	\onehalfspacing
	\maketitle
	\pagenumbering{gobble}
	\thispagestyle{empty}
	\tableofcontents
	\newpage
	\pagenumbering{arabic}
	
	\tikzset{
		mytext/.style={text height=1em},
		dot/.style={draw, fill, circle, minimum size=0.7ex, inner sep=0pt, outer sep=0pt}
	}
	
	\section{Uvod}
	Cilj ovog kratkog rada je upoznati čitatelje s osnovama i zainteresirati za daljnje proučavanje naglasnog sustava standardnog tokijskog dijalekta japanskog jezika.
	Japanski naglasni sustav je dosta različit od hrvatskog ili engleskog, stoga je bitno najprije upoznati se s temeljima tog sustava definiranim u ovom odjeljku.
	U idućem odjeljku dan je sustavni pregled tipova naglasaka kroz primjere u kojima se pojavljuju, uz poneko pravilo i najčešće iznimke koje se pojavljuju u svakodnevnom govoru.
	Ostali odjeljci obrađuju pravila vezana uz naglaske kod imenica, pridjeva i glagola.
	
	Pa, što bi Japanci rekli, よろしくおねがいします!
	
	\subsection{M\=ora}
	M\=ora je japanski ekvivalent sloga.
	To je ujedno i najmanja nedjeljiva izgovorna jedinica neke riječi.
	Karakteristično obilježje japanskog jezika jest da svaka m\=ora ima jednako trajanje.
	U načelu, 1 m\=ora je reprezentirana 1 hiraganom, osim malih ゃ, ゅ, ょ koji su nalijepljeni na drugu hiraganu, npr. u しゃ, しゅ, しょ.
	\textit{Samoglasno N} ん se broji kao 1 m\=ora.
	I \textit{sokuon} っ (npr. u げっかん) i \textit{ch\=oonpo} ー  (npr. u ハート) se broje kao 1 m\=ora.
	U tablici \ref{tab:mora-type} ispod kopiranoj s Wikipedije navedene su vrste m\=ora za radoznale:
	
	\begin{table}[htb]
		\centering
		\caption{Vrste m\=ora.}
		\label{tab:mora-type}
		\begin{tabular}{llll}
			\toprule
			& & & Br. m\=ora \\
			\multicolumn{2}{l}{Vrsta} & Primjer & (u primjeru) \\
			\midrule
			V & samoglasnici & \e{お} & 1 \\
			jV & palatalizirani samoglasnici & \e{よ} (世) & 1 \\
			CV & suglasnik i samoglasnik & \e{こ} (子) & 1 \\
			CjV & suglasnik i pal. samoglasnik & \e{きょ} (巨) & 1 \\
			R & produljenje samoglasnika & きょ\e{う} (今日)、キ\e{ー}  & 2 \\
			N & samoglasno N & こ\e{ん} (紺) & 2 \\
			Q & udvostručavanje suglasnika & こ\e{っ}こ (国庫) & 3 \\
			\bottomrule
		\end{tabular}
	\end{table}

	Određena riječ može imati naglasak na samo jednoj, ili nijednoj m\=ori.
	Od navedenih vrsta, samo V, jV, CV i CjV mogu nositi naglasak.
	
	\subsection{Tonovi, intonacija i naglasak}
	U japanskom tehnički razlikujemo samo dvije vrste tona: visoki i niski.
	Svaka m\=ora ima točno jedan od ta dva tona, ton se ne može mijenjati unutar iste m\=ore.
	Pri promjeni iz niskog tona u visoki ili obratno se osjeti razlika u visini; međutim, i među tonovima iste vrste visina također može blago varirati.
	Prirodno u izgovoru visina opada prema kraju naglasne cjeline.
	
	Intonacija je kretanje visine tonova u izgovorenoj riječi.
	Intonacija za sve riječi koje nemaju naglasak je ista --- neutralna, a za riječi koje naglasak imaju ovisi o mjestu gdje se naglasak nalazi.
	U japanskom jeziku neutralna intonacija je pomalo neobična: počinje s niskim tonom na prvoj m\=ori, a sve m\=ore poslije prve imaju visoki ton.
	Primjeri:
	\begin{multicols}{3}
		\begin{itemize}
			\item \pitch{に/0,ほ/1,ん/1,ご/1}(日本語)
			\item \pitch{こ/0,ん/1,に/1,ち/1,は/1}
			\item \pitch{く/0,る/1,ま/1}(車)
			\item \pitch{り/0,ん/1,ご/1}
			\item \pitch{し/0,ん/1,ぶ/1,ん/1}(新聞)
			\item \pitch{と/0,も/1,だ/1,ち/1}(友達)
		\end{itemize}
	\end{multicols}

	U japanskom jeziku naglasak se manifestira tako što nakon m\=ore koja je naglašena visina intonacije naglo padne, odnosno naglašena m\=ora je posljednja visoka u riječi.
	Primjeri (naglašena m\=ora je podebljana):
	\begin{multicols}{3}
		\begin{itemize}
			\item \pitch{\e{ね}/1,こ/0} (猫)
			\item \pitch{な/0,\e{が}/1,さ/0,き/0}(長崎)
			\item \pitch{に/0,ほ/1,ん/1,\e{じ}/1,ん/0}\\(日本人)
			\item \pitch{い/0,も/1,う/1,\e{と}/1}(妹)
			\item \pitch{コ/0,ー/1,\e{ヒ}/1,ー/0}
			\item \pitch{ピ/0,\e{カ}/1,チ\m{ュ}/0,ウ/0}
		\end{itemize}
	\end{multicols}
	
	\subsection{Naglasna cjelina}
	Naglasnu cjelinu čini niz riječi koji zajedno ima najviše jedan naglasak.
	Svaka riječ inherentno ima ili nema određeni naglasak, no kad se više riječi udruži u naglasnu cjelinu, naglasak u pojedinim riječima može se izgubiti ili pomaknuti prema određenim pravilima.
	Najčešće se radi o glavnoj riječi sa zanaglasnicama (čestice, kopula, pomoćni glagoli), ili o konjugiranom obliku neke riječi.
	Neki primjeri naglasnih cjelina (naglašena m\=ora, ako postoji, je podebljana):
	\begin{itemize}
		\item \pitch{こ/0,れ/1}+\pitch{は/0}=\pitch{こ/0,れ/1,は/1}
		\item \pitch{た/0,な/1,か/1}+\pitch{\e{で}/1,す/0}=\pitch{た/0,な/1,か/1,\e{で}/1,す/0}(田中です)
		\item \pitch{お/0,と/1,\e{こ}/1}+\pitch{\e{で}/1,す/0}=\pitch{お/0,と/1,\e{こ}/1,で/0,す/0}(男です)
		\item \pitch{\e{じ}/1,し\m{ょ}/0}+\pitch{\e{で}/1,す/0}=\pitch{\e{じ}/1,し\m{ょ}/0,で/0,す/0}(辞書です)
		\item \pitch{た/0,\e{べ}/1,る/0}+\pitch{さ/0,せ/1,る/1}+\pitch{ら/0,れ/1,る/1}
		=\pitch{た/0,べ/1,さ/1,せ/1,ら/1,\e{れ}/1,る/0}(食べさせられる)
		\item \pitch{あ/0,た/1,た/1,\e{か}/1,い/0}(暖かい)、\pitch{あ/0,た/1,\e{た}/1,か/0,く/0,て/0}(暖かくて)
	\end{itemize}
	
	\section{Naglasni tipovi}
	S obzirom na poziciju unutar riječi, naglaske možemo podijeliti u četiri kategorije:
	\begin{enumerate}
		\item \textit{heiban} 平板 --- neutralna intonacija, nema naglaska
		\item \textit{atamadaka} 頭高 --- naglasak je na prvoj m\=ori (glavi)
		\item \textit{nakadaka} 中高 --- naglasak nije na prvoj ni na zadnjoj m\=ori
		\item \textit{odaka} 尾高 --- naglasak je na zadnjoj m\=ori (repu)
	\end{enumerate}
	Iako možemo reći da se u stvari radi o samo \textit{dvije} kategorije (bez naglaska -- \textit{heiban}, i s naglaskom -- sve ostale), svaka od navedenih kategorija ima svoje posebnosti s obzirom na promjene riječi (konjugacije), stoga je te promjene lakše promatrati kroz 4 kategorije.
	
	Naglasci se u rječnicima još označavaju brojem prema m\=ori koja je naglašena.
	Tako se \textit{atamadaka} označava sa (1), \textit{nakadaka} brojevima od (2) nadalje, a \textit{odaka} brojem zadnje m\=ore u riječi ili, prema konvenciji negativnog indeksiranja\footnote{ovu konvenciju popularizirao je programski jezik Python, a radi se o brojanju elemenata niza unatrag, (-1) predstavlja zadnji, (-2) predzadnji element, itd.}, sa (-1).
	Heiban se označava sa (0).
	
	Kod samostalnih riječi, \textit{heiban} i \textit{odaka} se ne razlikuju, jer intonacija nikad ne padne ni u jednom ni u drugom slučaju.
	Međutim, kada te riječi čine naglasnu cijelinu s drugim riječima, razlika u intonaciji postane primjetljiva:
	\begin{itemize}
		\item \pitch{は/0,な/1}(鼻)、\pitch{は/0,な/1,が/1}(鼻が)--- \textit{heiban}
		\item \pitch{は/0,な/1}(花)、\pitch{は/0,な/1,が/0}(花が)--- \textit{odaka}
	\end{itemize}

	Također, možemo primijetiti kako su visine tonova prve dvije m\=ore unutar neke naglasne cjeline međusobno uvijek različite (bilo niska-visoka, ili visoka-niska), pa se po tome lako može prepoznati početak riječi ili naglasne cjeline.
	
	\subsection{Naglasni tipovi s obzirom na broj m\=ora u riječi}
	\subsubsection*{1 m\=ora}
	Riječi koje se sastoje od 1 m\=ore koje počinju niskom morom su \textit{heiban}, a one koje počinju visokom \textit{atamadaka}.
	Drugi naglasni tipovi nisu mogući za riječi s 1 m\=orom.
	Primjeri riječi različitih naglasnih tipova (skupa s česticom radi lakšeg razlikovanja):
	\begin{enumerate}
		\item \textit{heiban}
		\begin{multicols}{3}
			\begin{itemize}
				\item \pitch{き/0,が/1}(気が)
				\item \pitch{ひ/0,が/1}(日が)
				\item \pitch{は/0,が/1}(葉が)
			\end{itemize}
		\end{multicols}
		\item \textit{atamadaka}
		\begin{multicols}{3}
			\begin{itemize}
				\item \pitch{き/1,が/0}(木が)
				\item \pitch{ひ/1,が/0}(火が)
				\item \pitch{は/1,が/0}(歯が)
			\end{itemize}
		\end{multicols}
	\end{enumerate}
	
	\subsubsection*{2 m\=ore}
	Riječi koje se sastoje od 2 m\=ore mogu imati sve tipove osim \textit{nakadaka}.
	Primjeri:
	\begin{enumerate}
		\item \textit{heiban}
		\begin{multicols}{3}
			\begin{itemize}
				\item \pitch{く/0,ち/1,が/1}(口)
				\item \pitch{は/0,な/1,が/1}(鼻)
				\item \pitch{く/0,び/1,が/1}(首)
			\end{itemize}
		\end{multicols}
		\item \textit{atamadaka}
		\begin{multicols}{3}
			\begin{itemize}
				\item \pitch{か/1,た/0,が/0}(肩)
				\item \pitch{の/1,ど/0,が/0}(喉)
			\end{itemize}
		\end{multicols}
		\item \textit{odaka}
		\begin{multicols}{3}
			\begin{itemize}
				\item \pitch{か/0,み/1,が/0}(髪)
				\item \pitch{み/0,み/1,が/0}(耳)
				\item \pitch{む/0,ね/1,が/0}(胸)
				\item \pitch{は/0,ら/1,が/0}(腹)
				\item \pitch{ゆ/0,び/1,が/0}(指)
				\item \pitch{あ/0,し/1,が/0}(足)
			\end{itemize}
		\end{multicols}
	\end{enumerate}

	\subsubsection*{3 m\=ore}
	Tek riječi od 3 i više m\=ora mogu imati sve tipove naglasaka.
	One su najčešće \textit{heiban} i \textit{atamadaka}, a rjeđe \textit{nakadaka} ili \textit{odaka}, stoga je korisno zapamtiti česte riječi koje spadaju pod zadnje dvije kategorije.
	Primjeri:
	\begin{enumerate}
		\item \textit{heiban}
		\begin{multicols}{3}
			\begin{itemize}
				\item \pitch{ふ/0,し/1,ぎ/1,な/1}
				\item \pitch{カ/0,バ/1,ン/1,は/1}
				\item \pitch{か/0,ん/1,じ/1,は/1}(漢字は)
				\item \pitch{つ/0,ば/1,さ/1,は/1}(翼は)
				\item \pitch{う/0,わ/1,さ/1,は/1}(噂は)
				\item \pitch{し/0,か/1,た/1,は/1}(仕方は)
			\end{itemize}
		\end{multicols}
		\item \textit{atamadaka}
		\begin{multicols}{3}
			\begin{itemize}
				\item \pitch{ご/1,は/0,ん/0,は/0}(ご飯は)
				\item \pitch{ご/1,ぜ/0,ん/0,は/0}(午前は)
				\item \pitch{か/1,ぞ/0,く/0,は/0}(家族は)
				\item \pitch{じ/1,ん/0,じ\m{ゃ}/0,は/0}(神社は)
				\item \pitch{め/1,が/0,ね/0,は/0}
				\item \pitch{け/1,し/0,き/0,は/0}(景色は)
				\item \pitch{に/1,も/0,つ/0,は/0}(荷物は)
				\item \pitch{で/1,ん/0,き/0,は/0}(電気は)
				\item \pitch{て/1,ん/0,き/0,は/0}(天気は)
				\item \pitch{げ/1,ん/0,き/0,な/0}(元気な)
				\item \pitch{き/1,れ/0,い/0,な/0}
			\end{itemize}
		\end{multicols} \pagebreak %dirty
		\item \textit{nakadaka}
		\begin{multicols}{3}
			\begin{itemize}
				\item \pitch{す/0,こ/1,し/0,は/0}(少しは)
				\item \pitch{ひ/0,と/1,り/0,は/0}(一人は)
				\item \pitch{ひ/0,と/1,つ/0,は/0}(一つは)
				\item \pitch{い/0,つ/1,つ/0,は/0}(五つは)
				\item \pitch{き/0,の/1,う/0,は/0}(昨日は)
				\item \pitch{に/0,お/1,い/0,は/0}(匂いは)
				\item \pitch{あ/0,な/1,た/0,は/0} 
			\end{itemize}
		\end{multicols}
		\item \textit{odaka}
		\begin{multicols}{3}
			\begin{itemize}
				\item \pitch{あ/0,し/1,た/1,は/0}(明日は)
				\item \pitch{こ/0,と/1,ば/1,は/0}(言葉は)
				\item \pitch{ふ/0,た/1,り/1,は/0}(二人は)
				\item \pitch{ふ/0,た/1,つ/1,は/0}(二つは)
				\item \pitch{み/0,っ/1,つ/1,は/0}(三つは)
				\item \pitch{よ/0,っ/1,つ/1,は/0}(四つは)
				\item \pitch{み/0,ん/1,な/1,は/0}(皆は)
				\item \pitch{お/0,と/1,こ/1,は/0}(男は)
				\item \pitch{お/0,ん/1,な/1,は/0}(女は)
				\item \pitch{あ/0,た/1,ま/1,は/0}(頭は)
			\end{itemize}
		\end{multicols}
	\end{enumerate}
	
	\subsubsection*{4 m\=ore}
	Često su imenice od 4 m\=ore \textit{heiban}, pogotovo one koje se pišu s dva kanđija.
	Stoga opet valja zapamtiti česte iznimke od ovoga pravila.
	Primjeri: (ovaj put popratne čestice su izostavljene radi bolje preglednosti)
		\begin{enumerate}
		\item \textit{heiban}
		\begin{multicols}{3}
			\begin{itemize}
				\item \pitch{が/0,っ/1,こ/1,う/1}(学校)
				\item \pitch{ぎ/0,ん/1,こ/1,う/1}(銀行)
				\item \pitch{さ/0,い/1,て/1,い/1}(最低)
				\item \pitch{さ/0,い/1,あ/1,く/1}(最悪)
				\item \pitch{さ/0,い/1,こ/1,う/1}(最高)
				\item \pitch{へ/0,い/1,ば/1,ん/1}(平板)
				\item \pitch{べ/0,ん/1,き\m{ょ}/1,う/1}(勉強)
				\item \pitch{し/0,ん/1,ぱ/1,い/1}(心配)
			\end{itemize}
		\end{multicols}
		\item \textit{atamadaka}
		\begin{multicols}{3}
			\begin{itemize}
				\item \pitch{お/1,ん/0,が/0,く/0}(音楽)
				\item \pitch{あ/1,い/0,さ/0,つ/0}(挨拶)
				\item \pitch{め/1,い/0,わ/0,く/0}(迷惑)
				\item \pitch{ま/1,い/0,に/0,ち/0}(毎日)
				\item \pitch{ま/1,い/0,あ/0,さ/0}(毎朝)
				\item \pitch{ま/1,い/0,ば/0,ん/0}(毎晩)
			\end{itemize}
		\end{multicols} \pagebreak %dirty
		\item \textit{nakadaka}
		\begin{multicols}{3}
			\begin{itemize}
				\item \pitch{せ/0,ん/1,せ/1,い/0}(先生)
				\item \pitch{は/0,ん/1,ぶ/1,ん/0}(半分)
				\item \pitch{あ/0,ん/1,な/1,い/0}(案内)
				\item \pitch{お/0,と/1,と/1,い/0}(一昨日)
				\item \pitch{ア/0,パ/1,ー/0,ト/0}
				\item \pitch{し/0,つ/1,れ/0,い/0}(失礼)
				\item \pitch{の/0,み/1,も/0,の/0}(飲み物)
			\end{itemize}
		\end{multicols}
		\item \textit{odaka}
		\begin{multicols}{2}
			\begin{itemize}
				\item \pitch{い/0,も/1,う/1,と/1}(妹)
				\item \pitch{お/0,と/1,う/1,と/1}(弟)
			\end{itemize}
		\end{multicols}
	\end{enumerate}
	
	\subsection{Još par korisnih pravila}
	Dva su korisna pravila vezana uz zamjenice:
	\begin{enumerate}
		\item Zamjenice koje počinju na k-, s-, a- su \textit{heiban}:
			\begin{multicols}{3}
				\begin{itemize} 
					\item \pitch{こ/0,れ/1}
					\item \pitch{そ/0,こ/1}
					\item \pitch{あ/0,の/1}
					\item \pitch{そ/0,う/1}
					\item \pitch{こ/0,ち/1,ら/1}
					\item itd.
				\end{itemize}
			\end{multicols}
		\item Upitne zamjenice su \textit{atamadaka}:
			\begin{multicols}{5}
				\begin{itemize}
					\item \pitch{ど/1,れ/0}
					\item \pitch{ど/1,こ/0}
					\item \pitch{ど/1,の/0}
					\item \pitch{ど/1,う/0}
					\item \pitch{ど/1,ち/0,ら/0}
					\item \pitch{だ/1,れ/0}
					\item \pitch{い/1,つ/0}
					\item \pitch{な/1,に/0}
					\item \pitch{な/1,ぜ/0}
					\item itd.
				\end{itemize}
			\end{multicols}
	\end{enumerate}

	Iznimno, uz uobičajenu verziju zamjenice そう postoji još jedna koja se više upotrebljava kao imenica, a ona je \textit{atamadaka}.
	Prva je priložna, pa ona opisuje glagole:
	\begin{itemize}
		\begin{multicols}{3}
			\item \pitch{そ/0,う/1,い/1,う/1}
			\item \pitch{そ/0,う/1,す/1,る/1}
			\item \pitch{そ/0,う/1,し/1,よ/1,う/1}
		\end{multicols}
	\end{itemize}
	Druga se koristi kao imenica (sufiksna imenica), nju prepoznajemo po tome što iza nje može doći kopula:
	\begin{itemize}
		\begin{multicols}{3}
			\item \pitch{そ/1,う/0,で/0,す/0,か/0}?
			\item \pitch{そ/1,う/0,だ/0,な/0}…
			\item \pitch{ふ/0,り/1,そ/1,う/0,で/0,す/0}\\(降りそうです\footnote{više o promjeni naglasaka kod glagola kasnije})
		\end{multicols}
	\end{itemize}
	

	\section{Naglasak kod imenica}
	Imenice mogu imati bilo koji od 4 naglaska, kao što je pokazano na nizu primjera u prethodnom odjeljku.
	Specijalni slučaj čine dugačke imenice koje su nastale srastanjem dviju ili više drugih imenica.
	Njihov naglasni tip je gotovo uvijek \textit{nakadaka}, i to takav da je naglasak na prvoj mori druge (ili posljednje) riječi u nizu.
	Pritom nije bitno kakav naglasak imaju individualne riječi.
	
	Primjeri:
	\begin{itemize}
		\item \pitch{や/0,ま/1,が/0}(山が)+\pitch{の/0,ぼ/1,り/1,が/1}(登りが)=\pitch{や/0,ま/1,の/1,ぼ/0,り/0}(山登り)
		\item \pitch{は/0,く/1,じ/1,が/1}(白人が)+\pitch{だ/0,ん/1,せ/1,い/1,が/1}(男性が)=%
		\begin{minipage}{3cm}\pitch{は/0,く/1,じ/1,ん/1,だ/0,ん/0,せ/0,い/0}\\(白人男性)\end{minipage}
	\end{itemize}
	
	Međutim, ima i riječi koje ne slijede gore navedeno pravilo o naglasku na početku druge riječi.
	Riječi za nacionalnost sa sufiksom 人 uglavnom imaju naglasak jednu m\=oru prije じ iz 人, osim u slučaju kada ta m\=ora ne može nositi naglasak (日本, スペイン).
	Riječi za jezik sa sufiksom 語 su \textit{heiban} bez obzira na naglasak prve riječi.
	U tablici \ref{tab:drzava} navedeni su neki primjeri država, nacionalnosti i jezika.
	
	\begin{table}[htbp]
		\begin{minipage}{\textwidth}
			\renewcommand{\arraystretch}{2}
			\centering
			\caption[]{Primjeri država, jezika i nacionalnosti\footnote{popunjena s čim više primjera da stranica ne zjapi prazna}}
			\label{tab:drzava}
			\begin{tabular}{llll}
				\toprule
				& \multicolumn{1}{c}{国} & \multicolumn{1}{c}{人} & \multicolumn{1}{c}{語} \\
				\midrule
				日本 & \pitch{に/0,ほ/1,ん/0} & \pitch{に/0,ほ/1,ん/1,じ/1,ん/0} & \pitch{に/0,ほ/1,ん/1,ご/1} \\
				中国 & \pitch{ち\m{ゅ}/1,う/0,ご/0,く/0} & \pitch{ち\m{ゅ}/0,う/1,ご/1,く/1,じ/0,ん/0} & \pitch{ち\m{ゅ}/0,う/1,ご/1,く/1,ご/1} \\
				韓国 & \pitch{か/1,ん/0,こ/0,く/0} & \pitch{か/0,ん/1,こ/1,く/1,じ/0,ん/0} & \pitch{か/0,ん/1,こ/1,く/1,ご/1} \\
				ロシア & \pitch{ろ/1,し/0,あ/0} & \pitch{ろ/0,し/1,あ/1,じ/0,ん/0} & \pitch{ろ/0,し/1,あ/1,ご/1} \\
				ドイツ & \pitch{ど/1,い/0,つ/0} & \pitch{ど/0,い/1,つ/1,じ/0,ん/0} & \pitch{ど/0,い/1,つ/1,ご/1} \\
				スペイン & \pitch{す/0,ぺ/1,い/0,ん/0} & \pitch{す/0,ぺ/1,い/1,ん/0,じ/0,ん/0} & \pitch{す/0,ぺ/1,い/1,ん/1,ご/1} \\
				フランス & \pitch{ふ/0,ら/1,ん/1,す/1} & \pitch{ふ/0,ら/1,ん/1,す/1,じ/0,ん/0} & \pitch{ふ/0,ら/1,ん/1,す/1,ご/1} \\
				ポーランド & \pitch{ぽ/1,ー/0,ら/0,ん/0,ど/0} & \pitch{ぽ/0,ー/1,ら/1,ん/1,ど/1,じ/0,ん/0} & \pitch{ぽ/0,ー/1,ら/1,ん/1,ど/1,ご/1} \\
				クロアチア & \pitch{く/0,ろ/1,あ/1,ち/1,あ/1} & \pitch{く/0,ろ/1,あ/1,ち/1,あ/1,じ/0,ん/0} & \pitch{く/0,ろ/1,あ/1,ち/1,あ/1,ご/1} \\
				セルビア & \pitch{せ/0,る/1,び/1,あ/1} & \pitch{せ/0,る/1,び/1,あ/1,じ/0,ん/0} & \pitch{せ/0,る/1,び/1,あ/1,ご/1} \\
				ボスニア & \pitch{ぼ/0,す/1,に/1,あ/1} & \pitch{ぼ/0,す/1,に/1,あ/1,じ/0,ん/0} & \pitch{ぼ/0,す/1,に/1,あ/1,ご/1} \\
				アラビア & \pitch{あ/0,ら/1,び/1,あ/1} & \pitch{あ/0,ら/1,び/1,あ/1,じ/0,ん/0} & \pitch{あ/0,ら/1,び/1,あ/1,ご/1} \\
				\makecell[tl]{ナメック星\\(Namek)} & \pitch{な/0,め/1,っ/1,く/1,せ/1,い/1} & \pitch{な/0,め/1,っ/1,く/1,せ/1,い/0,じ/0,ん/0} & \pitch{な/0,め/1,く/1,ご/1} \\
				\makecell[tl]{クロノシュ\\(Q'onoS)} & \pitch{く/0,ろ/1,の/1,し\m{ゅ}/1} & \pitch{く/0,り/1,ん/1,ご/1,ん/0,じ/0,ん/0} & \pitch{く/0,り/1,ん/1,ご/1,ん/1,ご/1} \\
				\bottomrule
			\end{tabular}
			\renewcommand{\arraystretch}{1.25}
		\end{minipage}
	\end{table}

	
	\subsection{Naglasak u čestim kombinacijama imenica s drugim riječima}
	Dosad razmatrani primjeri uključivali su po jednu jednostavnu ili složenu imenicu koja sama za sebe čini naglasnu cjelinu.
	No, imenice se spajaju i s kopulama i različitim sufiksima, čineći opet jednu naglasnu cjelinu koja ima promijenjen naglasak u odnosu na individualne riječi od kojih je nastala.
	Nasreću, ove promjene se doagađaju po predvidivim, relativno jednostavnim pravilima.
	
	\subsubsection*{Spajanje imenica s kopulama だ i です}
	Spajanjem imenice s kopulom だ nastaje naglasna cjelina čiji naglasak ovisi o naglasku imenice, na sljedeći način:
	\begin{enumerate}
		\item Ako imenica nema naglasak (\textit{heiban}), tada naglasna cjelina opet nema naglasak
		\begin{itemize}
			\item \pitch{が/0,く/1,せ/1,い/1}+\pitch{だ/0}=\pitch{が/0,く/1,せ/1,い/1,だ/1}(学生)
		\end{itemize}
		\item Ako imenica ima naglasak (ostali tipovi), tada naglasna cjelina ima naglasak na istom mjestu na kojem je bio u imenici
		\begin{itemize}
			\item \textit{atamadaka} --- \pitch{じ/1,し\m{ょ}/0}+\pitch{だ/0}=\pitch{じ/1,し\m{ょ}/0,だ/0}(辞書)
			\item \textit{nakadaka} --- \pitch{せ/1,ん/1,せ/1,い/0}+\pitch{だ/0}=\pitch{せ/0,ん/1,せ/1,い/0,だ/0}(先生)
			\item \textit{odaka} --- \pitch{お/0,と/1,こ/1}+\pitch{だ/0}=\pitch{お/0,と/1,こ/1,だ/0}(男)
		\end{itemize}
	\end{enumerate}

	S druge strane, spajanje imenice s kopulom です uvodi male promjene u naglasak:
	\begin{enumerate}
		\item Ako imenica nema naglasak (\textit{heiban}), tada naglasna cjelina ima naglasak na mori で u です:
		\begin{itemize}
			\item \pitch{が/0,く/1,せ/1,い/1}+\pitch{で/1,す/0}=\pitch{が/0,く/1,せ/1,い/1,で/1,す/0}(学生)
		\end{itemize}
		\item Ako imenica ima naglasak (ostali tipovi), tada naglasna cjelina ima naglasak na istom mjestu na kojem je bio u imenici
		\begin{itemize}
			\item \textit{atamadaka} --- \pitch{じ/1,し\m{ょ}/0}+\pitch{で/1,す/0}=\pitch{じ/1,し\m{ょ}/0,で/0,す/0}(辞書)
			\item \textit{nakadaka} --- \pitch{せ/1,ん/1,せ/1,い/0}+\pitch{で/1,す/0}=\pitch{せ/0,ん/1,せ/1,い/0,で/0,す/0}(先生)
			\item \textit{odaka} --- \pitch{お/0,と/1,こ/1}+\pitch{で/1,す/0}=\pitch{お/0,と/1,こ/1,で/0,す/0}(男)
		\end{itemize}
	\end{enumerate}

	Iako se kopule \pitch{だ/0} i \pitch{で/1,す/0} ne koriste kao samostalne riječi, u svrhu njihovog spajanja na imenice možemo si pomoći ako ih promatramo kao \textit{heiban}, odnosno \textit{atamadaka}.
	Tada se gornje promjene mogu opisati samo jednim pravilom (koje vrijedi i šire od spajanja imenica s kopulom): \textit{atamadaka}, \textit{nakadaka} i \textit{odaka} \textit{gaze}\footnote{engl. \textit{override}, termin posuđen iz koncepta objektno orijentiranog programiranja koji u današnje doba, kada jezici kao što je Python dobivaju na popularnosti i rugaju se sa smislom života, doživljava sve veću propast} \textit{heiban}.
	Možemo to promatrati i na još jedan način: \textit{heiban} je u stvari odsutnost naglaska, a budući da naglasna cjelina može imati najviše jedan naglasak, ostaje onaj naglasak koji je došao prvi po redu.
	{\footnotesize\color{white} \textit{Treći način (hard-core): skup naglasaka, asocijativna operacija 
	$f(\text{平板}, b) = b,\ f\left(a, \color{gray}\cdot\color{white}\right) = a$, i neutralni element 平板 čine monoid.}}

	\subsubsection*{Spajanje imenica s česticama}
	Često korištene čestice od jedne m\=ore, \pitch{は/0}、\pitch{も/0}、\pitch{が/0}、\pitch{を/0}、\pitch{に/0}、\pitch{で/0}、\pitch{と/0} nikad se ne koriste samostalno, ali opet, u okviru njihovog povezivanja na imenice, možemo smatrati da su \textit{heiban}.
	U tom slučaju naglasak će ostati na istom mjestu nakon što se na imenicu nalijepi neka čestica.
	
	Zapravo, ranije je razlika između \textit{heiban} i \textit{odaka} naglasnih tipova demonstrirana dodavanjem ovih čestica na kraj riječi upravo zahvaljujući ovom svojstvu, a to je da one ne mijenjaju mjesto naglaska unutar naglasne cijeline.
	
	Na isti način, složene čestice \pitch{で/1,は/0}、\pitch{で/1,も/0}、\pitch{に/1,も/0}、\pitch{と/1,も/0}、\pitch{か/1,が/0}、\pitch{か/1,を/0}、\pitch{か/1,も/0} možemo promatrati kao da su \textit{odaka}.
	Pravila o \textit{gaženju} naglasaka ostaju ista.
	
	\subsubsection*{Spajanje imenica s ostalim čestim nastavcima}
	Svršeni oblici kopule \pitch{だ/1,っ/0,た/0} i \pitch{で/1,し/0,た/0}, volicionalni oblici \pitch{だ/0,ろ/1,う/0} i \pitch{で/0,し\m{ょ}/1,う/0}, te hipotetski oblik \pitch{な/1,ら/0,ば/0} mogu se uzeti kao da imaju naglasak kako im je naznačeno.
	Pri spajanju s \textit{heiban} riječima cjelina preuzima naglasak kopule, a pri spajanju s ostalim tipovima naglasaka naglasak ostaje na riječi s kojom se spaja.
	Primjeri povezivanja sa だろう i でし\!ょう (na isti način se povezuju i だった、でした、ならば):
	\begin{itemize}
		\item \textit{heiban} --- \pitch{が/0,く/1,せ/1,い/1,だ/1,ろ/1,う/0}、\pitch{が/0,く/1,せ/1,い/1,で/1,し\m{ょ}/1,う/0}(学生)
		\item \textit{atamadaka} --- \pitch{じ/1,し\m{ょ}/0,だ/0,ろ/0,う/0}、\pitch{じ/1,し\m{ょ}/0,で/0,し\m{ょ}/0,う/0}(辞書)
		\item \textit{nakadaka} --- \pitch{せ/0,ん/1,せ/1,い/0,だ/0,ろ/0,う/0}、\pitch{せ/0,ん/1,せ/1,い/0,で/0,し\m{ょ}/0,う/0}(先生)
		\item \textit{odaka} --- \pitch{お/0,と/1,こ/1,だ/0,ろ/0,う/0}、\pitch{お/0,と/1,こ/1,で/0,し\m{ょ}/0,う/0}(男)
	\end{itemize}
	
	Također, često se koriste i \pitch{み/1,た/0,い/0} te \pitch{ら/0,し/1,い/0} kada znači da se radi o govornikovom zaključku temeljenom na nečemu što je čuo, vidio, ili pročitao.
	Vrijede ista pravila spajanja kao i za dosad navedeno:
	\begin{itemize}
		\item \textit{heiban} --- \pitch{が/0,く/1,せ/1,い/1,み/1,た/0,い/0}、\pitch{が/0,く/1,せ/1,い/1,ら/1,し/1,い/0}(学生)
		\item \textit{atamadaka} --- \pitch{じ/1,し\m{ょ}/0,み/0,た/0,い/0}、\pitch{じ/1,し\m{ょ}/0,ら/0,し/0,い/0}(辞書)
		\item \textit{nakadaka} --- \pitch{せ/0,ん/1,せ/1,い/0,み/0,た/0,い/0}、\pitch{せ/0,ん/1,せ/1,い/0,ら/0,し/0,い/0}(先生)
		\item \textit{odaka} --- \pitch{お/0,と/1,こ/1,み/0,た/0,い/0}、\pitch{お/0,と/1,こ/1,ら/0,し/0,い/0}(男)
	\end{itemize}
	
	Jedna često korištena iznimka je sufiks \pitch{ら/0,し/1,い/0} kada znači reprezentativnu karakteristiku nečeg\footnote{
	razlikujemo ga od prvog らしい po tome što prvi završava rečenicu (moguće i opisnu rečenicu) i mijenja njezin modalitet, a drugi je sufiks koji se dodaje na imenicu i ponaša se kao idjev pa može biti ispred imenice, ali prvi se isto ponaša kao idjev, znači ne razlikujemo ih}.
	Razlikovati ih možemo jedino po smislu.
	Pravilo da naglasna cjelina ima najviše jedan naglasak i dalje vrijedi, međutim ovaj sufiks uvijek gazi naglasak prve riječi:
	\begin{itemize}
		\item \textit{heiban} --- \pitch{が/0,く/1,せ/1,い/1,ら/1,し/1,い/0}(学生)
		\item \textit{atamadaka} --- \pitch{じ/0,し\m{ょ}/1,ら/1,し/1,い/0}(辞書)
		\item \textit{nakadaka} --- \pitch{せ/0,ん/1,せ/1,い/1,ら/1,し/1,い/0}(先生)
		\item \textit{odaka} --- \pitch{お/0,と/1,こ/1,ら/1,し/1,い/0}(男)
	\end{itemize}
	Primijetimo da, u slučaju spajanja sa \textit{heiban} riječima, između prvog i drugog oblika nema razlike u naglasku.
	
	Na isti način se ponašaju i nastavci \pitch{っ/0,ぽ/1,い/0} i \pitch{ふ/0,う/1}(風), tj. njihovi naglasci uvijek gaze naglasak prve riječi.
	
	\section{Naglasak kod pridjeva}
	Pridjevi se u japanskom jeziku dijele u dvije vrste: neprave, imenske, na-pridjeve ili skraćeno --- nadjeve, te prave, i-pridjeve ili skraćeno --- idjeve.
	Uloga obiju vrsta je opisivanje imenice, ali ove vrste se gramatički vrlo različito ponašaju.
	Stoga će i njihovi naglasci biti različiti.
	
	\subsection{Naglasak kod osnovnih oblika nadjeva}
	Nadjevi su u osnovi podvrsta imenica s privilegijom da se mogu koristiti za opisivanje drugih imenica.
	To znači da sve što je do sada vrijedilo za imenice vrijedi i za nadjeve.
	Dodatno, nadjevi se spajaju sa opisnim oblikom kopule \pitch{な/0} na jednak način kao i imenice sa završnim oblik kopule \pitch{だ/0}, zajedno tvoreći naglasnu cjelinu.
	Imenica koju nadjev opisuje pripada zasebnoj naglasnoj cjelini.
	Iznimno, ako je naglasna cjelina pridjeva nenaglašena (\textit{heiban}), tada se može združiti s imenicom u novu naglasnu cjelinu koja ima naglasak na istom mjestu kao i ta imenica.
	Primjeri:
	\begin{itemize}
		\item \pitch{す/0,て/1,き/1}+\pitch{な/1}+\pitch{せ/0,ん/1,せ/1,い/0}%
		\begin{minipage}[t]{8cm}
			=\pitch{す/0,て/1,き/1,な/1}\!\pitch{せ/0,ん/1,せ/1,い/0}(素敵な先生)\\
			\hspace*{0em}(\pitch{す/0,て/1,き/1,な/1,せ/1,ん/1,せ/1,い/0})
		\end{minipage}
		\item \pitch{げ/1,ん/0,き/0}+\pitch{な/0}+\pitch{が/0,く/1,せ/1,い/1}=\pitch{げ/1,ん/0,き/0,な/0}\!\pitch{が/0,く/1,せ/1,い/1}(元気な学生)
		\item \pitch{じ/0,ゆ/1,う/0}+\pitch{な/0}+\pitch{お/0,と/1,こ/1}=\pitch{じ/0,ゆ/1,う/0,な/0}\!\pitch{お/0,と/1,こ/1}(自由な男)
		\item \pitch{す/0,き/1,な/0}+\pitch{な/0}+\pitch{じ/1,し\m{ょ}/0}=\pitch{す/0,き/1,な/0}\!\pitch{じ/1,し\m{ょ}/0}(好きな辞書)
	\end{itemize}

	Tzv. prošli oblik nadjeva dobiva se spajanjem sa \pitch{だ/1,っ/0,た/0}.
	Pritom nastala naglasna cjelina nikad nije \textit{heiban}, jer je だった naglašena.
	
	Pravilo o riječima od 4 m\=ore koje se pišu s 2 kanđija vrijedi i za nadjeve.
	Neke češće iznimke koje bi vrijedilo zapamtiti su:
	\begin{itemize}
		\begin{multicols}{2}
			\item \pitch{て/1,い/0,ね/0,い/0,な/0}(丁寧な)
			\item \pitch{ね/1,っ/0,し/0,ん/0,な/0}(熱心な)
			\item \pitch{じゅ/0,う/1,ぶ/1,ん/0,な/0}(十分な)
			\item \pitch{し/1,ん/0,せ/0,つ/0,な/0,/0}(親切な)
		\end{multicols}
	\end{itemize}

	Evo i primjera učestalih nadjeva različitih naglasnih tipova:
	\begin{enumerate}
		\item \textit{heiban}
		\begin{itemize}
			\begin{multicols}{3}
				\item \pitch{ひ/0,ま/1,な/1}(暇な)
				\item \pitch{む/0,だ/1,な/1}(無駄な)
				\item \pitch{す/0,て/1,き/1,な/1}(素敵な)
				\item \pitch{き/0,ら/1,い/1,な/1}(嫌いな)
				\item \pitch{じ\m{ょ}/0,う/1,ぶ/1,な/1}(丈夫な)
			\end{multicols}
		\end{itemize}
		\item \textit{atamadaka}
		\begin{itemize}
			\begin{multicols}{3}
				\item \pitch{へ/1,ん/0,な/0}(変な)
				\item \pitch{む/1,り/0,な/0}(無理な)
				\item \pitch{げ/1,ん/0,き/0,な/0}(元気な)
			\end{multicols}
			\begin{multicols}{3}
				\item \pitch{べ/1,ん/0,り/0,な/0}(便利な)
				\item \pitch{ぶ/1,べ/0,ん/0,な/0}(不便な)
				\item \pitch{し/1,ず/0,か/0,な/0}(静かな)			
			\end{multicols}
			\item \pitch{だ/1,い/0,す/0,き/0,な/0}\\(大好きな)
		\end{itemize}
		\item \textit{nakadaka}
		\begin{itemize}
			\begin{multicols}{2}
				\item \pitch{じ/0,ゆ/1,う/0,な/0}(自由な)
				\item \pitch{だ/0,い/1,じ\m{ょ}/1,う/0,ぶ/0,な/0,/0}(大丈夫な)
			\end{multicols}
		\end{itemize}
		\item \textit{odaka}
		\begin{itemize}
			\begin{multicols}{3}
				\item \pitch{す/0,き/1,な/0}(好きな)
				\item \pitch{い/0,や/1,な/0}(嫌な)
				\item \pitch{ら/0,く/1,な/0}(楽な)
				\item \pitch{じ\m{ょ}/0,う/1,ず/1,な/0}(上手な)
				\item \pitch{へ/0,た/1,な/0}(下手な)
				\item \pitch{だ/0,め/1,な/0}(だめな)
			\end{multicols}
		\end{itemize}
	\end{enumerate}

	Spajanje nadjeva sa さ nije toliko uobičajeno, već se umjesto koriste druge riječi (npr. umjesto \ruby{上手}{じょうず}さ koristi se \ruby{上手}{うま}さ) ili te riječi imaju nepravilne oblike (npr. \ruby{静}{しず}か --- \ruby{静}{しず}けさ).
	Pravilo kod さ oblika nadjeva je da naglasak dolazi 1 m\=oru prije さ, osim tamo gdje ta m\=ora ne može nositi naglasak (tada 2 m\=ore prije さ), te u \textit{heiban} nadjevima, gdje naglasak ostaje nepromijenjen (\textit{heiban}).
	Tako imamo:
	\begin{itemize}
		\item \pitch{べ/1,ん/0,り/0,な/0}(便利な)、\pitch{べ/0,ん/1,り/1,さ/0}(便利さ)
		\item \pitch{じ/0,ゆ/1,う/0,な/0}(自由な)、\pitch{じ/0,ゆ/1,う/0,さ/0}(自由さ)
		\item \pitch{き/0,ら/1,い/1,な/1}(嫌いな)、\pitch{き/0,ら/1,い/1,さ/1}(嫌いさ)
	\end{itemize}
	
	\subsection{Naglasak kod osnovnih oblika idjeva}
	Idjevi su, za razliku od nadjeva, pravi pridjevi, odnosno punokrvna vrsta riječi.
	Razlikujemo ih od nadjeva po tom što imaju nastavak い koji se također mijenja ovisno o gramatičkom obliku, dok u slučaju nadjeva samo kopula ta koja se mijenja i ona nije sastavni dio riječi.
	
	Idjevi eksponiraju vrlo osebujne uzorke naglasnih tipova, stoga je njihova temeljita kategorizacija dobra početna točka kod učenja njihovih naglasaka.
	
	\subsubsection*{\textit{Atamadaka} idjevi}
	U suvremenom japanskom jeziku ima jako malo \textit{atamadaka} idjeva.
	Među onima koji se često susreću u svakodnevnom govoru su:
	\begin{itemize}
		\begin{multicols}{3}
			\item \pitch{よ/1,い/0}、\pitch{い/1,い/0}(良い)
			\item \pitch{こ/1,い/0}(濃い)
			\item \pitch{お/1,お/0,い/0}(多い)
		\end{multicols}
	\end{itemize}
	Oni su najbolji od svih idjeva zato što u izvedenim oblicima nikad ne mijenjaju naglasak.
	Šteta su samo tri.
	Zapravo, ima još jedan, nepostojeći.
	
	\subsubsection*{\textit{Nakadaka} idjevi}
	\textit{Nakadaka} je najčešći naglasni tip među idjevima.
	Važno je također da svi \textit{nakadaka} idjevi naglasak imaju točno na (-2) m\=ori, tj. jednoj prije završnog い.
	Kako je ovo najbrojnija kategorija, vrijedi naučiti napamet one idjeve koje ne pripadaju ovoj kategoriji, te za ostale pretpostaviti da su \textit{nakadaka}.
	Primjeri:
	\begin{itemize}
		\begin{multicols}{3}
			\item \pitch{あ/0,つ/1,い/0}(暑い、熱い)
			\item \pitch{か/0,わ/1,い/1,い/0}(かわいい)
			\item \pitch{お/0,も/1,し/1,ろ/1,い/0}(面白い)
		\end{multicols}
	\end{itemize}
	
	\subsubsection*{\textit{Odaka} idjevi}
	\textit{Odaka} idjevi, koji bi naglasak imali na posljednjoj mori, い, ne postoje.
	Oni bi možda bili još bolji od \textit{atamadaka} idjeva, ali šteta što ne postoje.
	
	\subsubsection*{\textit{Heiban} idjevi}
	Poseban žulj među ovim kategorijama predstavljaju, naravno, \textit{heiban} idjevi.
	Neki od učestalih u svakodnevnom govoru su:
	\begin{itemize}
		\begin{multicols}{3}
			\item \pitch{あ/0,か/1,い/1}(赤い)
			\item \pitch{あ/0,ま/1,い/1}(甘い)
			\item \pitch{お/0,も/1,い/1}(重い)
			\item \pitch{か/0,る/1,い/1}(軽い)
			\item \pitch{く/0,ら/1,い/1}(暗い)
			\item \pitch{あ/0,か/1,る/1,い/1}(明るい)
			\item \pitch{つ/0,め/1,た/1,い/1}(冷たい)
			\item \pitch{き/0,い/1,ろ/1,い/1}(黄色い)
			\item \pitch{む/0,ず/1,か/1,し/1,い/1}(難しい)
		\end{multicols}
	\end{itemize}
	No, \textit{heiban} idjevi se danas sve češće izgovaraju kao da su \textit{nakadaka} (s naglaskom na (-2) m\=ori).
	Dakle, može i sljedeće:
	\begin{itemize}
		\begin{multicols}{3}
			\item \pitch{あ/0,か/1,い/0}(赤い)
			\item \pitch{あ/0,ま/1,い/0}(甘い)
			\item \pitch{お/0,も/1,い/0}(重い)
			\item \pitch{か/0,る/1,い/0}(軽い)
			\item \pitch{く/0,ら/1,い/0}(暗い)
			\item \pitch{あ/0,か/1,る/1,い/0}(明るい)
			\item \pitch{つ/0,め/1,た/1,い/0}(冷たい)
			\item \pitch{き/0,い/1,ろ/1,い/0}(黄色い)
			\item \pitch{む/0,ず/1,か/1,し/1,い/0}(難しい)
		\end{multicols}
	\end{itemize}

	Međutim, prava prijevara leži u tome da se ova iznimka odnosi samo na osnovni oblik; svi izvedeni oblici ponašaju se prema pravilima specifičnim za \textit{heiban} naglasni tip.
	
	\subsection{Naglasak kod izvedenih oblika idjeva}
	Kako je prethodno spomenuto, idjevi završavaju nastavkom い koji se mijenja ovisno o gramatičkom obliku, te, za razliku od nadjeva, u idjevima se pritom može promijeniti i mjesto naglaska.
	Evo glavnih izvedenih oblika i njihovih naglasaka s obzirom na prije navedene kategorije:
	
	\subsubsection*{Izvedeni oblici kod \textit{atamadaka} idjeva}
	Ranije je spomenuto da \textit{atamadaka} riječi ostaju \textit{atamadaka} i u izvedenim oblicima.
	Evo i primjera:
	\begin{itemize}
		\item \pitch{よ/1,い/0}、\pitch{よ/1,く/0}、\pitch{よ/1,さ/0}、\pitch{よ/1,く/0,て/0}、\pitch{よ/1,か/0,っ/0,た/0}、\pitch{よ/1,け/0,れ/0,ば/0}(良い)
		\item \pitch{こ/1,い/0}、\pitch{こ/1,く/0}、\pitch{こ/1,さ/0}、\pitch{こ/1,く/0,て/0}、\pitch{こ/1,か/0,っ/0,た/0}、\pitch{こ/1,け/0,れ/0,ば/0}(濃い)
		\item \pitch{お/1,お/0,い/0}、\pitch{お/1,お/0,く/0}、\pitch{お/1,お/0,さ/0}、\pitch{お/1,お/0,く/0,て/0}、\pitch{お/1,お/0,か/0,っ/0,た/0}、\pitch{お/1,お/0,け/0,れ/0,ば/0}\\(多い)
	\end{itemize}
	
	\subsubsection*{Izvedeni oblici kod \textit{nakadaka} idjeva}
	\textit{Nakadaka} idjevi se također gotovo skroz pravilno mijenjaju.
	Postoji samo jedna iznimka, i to kod idjeva od 3 m\=ore: u izvedenim oblicima naglasak se s 2. m\=ore prebacuje na 1., odnosno oni postaju \textit{atamadaka}.
	Primjer:
	\begin{itemize}
		\item
		\pitch{あ/0,つ/1,い/0}: 
		\pitch{あ/1,つ/0,く/0}、
		\pitch{あ/1,つ/0,さ/0}、
		\pitch{あ/1,つ/0,く/0,て/0}、
		\pitch{あ/1,つ/0,か/0,っ/0,た/0}、
		\pitch{あ/1,つ/0,け/0,れ/0,ば/0}\\(暑い)
	\end{itemize}

	\textit{Nakadaka} idjevi s 4 ili više m\=ora zadržavaju naglasak na istom mjestu kao i u osnovnom obliku:
	\begin{itemize}
		\item
		\pitch{か/0,わ/1,い/1,い/0}: 
		\pitch{か/0,わ/1,い/1,く/0}、
		\pitch{か/0,わ/1,い/1,さ/0}、
		\pitch{か/0,わ/1,い/1,く/0,て/0}、
		\pitch{か/0,わ/1,い/1,か/0,っ/0,た/0}、\\
		\pitch{か/0,わ/1,い/1,け/0,れ/0,ば/0}(可愛い)
		\item
		\pitch{お/0,も/1,し/1,ろ/1,い/0}: 
		\pitch{お/0,も/1,し/1,ろ/1,く/0}、
		\pitch{お/0,も/1,し/1,ろ/1,さ/0}、
		\pitch{お/0,も/1,し/1,ろ/1,く/0,て/0}、\\
		\pitch{お/0,も/1,し/1,ろ/1,か/0,っ/0,た/0}、
		\pitch{お/0,も/1,し/1,ろ/1,け/0,れ/0,ば/0}(面白い)
	\end{itemize}

	\subsubsection*{Izvedeni oblici kod \textit{heiban} idjeva}
	Kako je prethodno spomenuto, \textit{heiban} idjevi u svom osnovnom obliku mogu se još izgovarati i kao \textit{nakadaka}.
	Izvedeni oblici ponašaju se prema posebnim pravilima za \textit{heiban} idjeve.
	Tako 〜く i 〜さ oblici ostaju \textit{heiban}, dok 〜くて, 〜かった i 〜ければ imaju naglasak na jednoj mori prije い u osnovnom obliku (kao da su \textit{nakadaka} idjevi).
	Primjeri:
	\begin{itemize}
		\item \pitch{あ/0,か/1,い/1}: 
		\pitch{あ/0,か/1,く/1}
		\pitch{あ/0,か/1,さ/1}、
		\pitch{あ/0,か/1,く/0,て/0}、	
		\pitch{あ/0,か/1,か/0,っ/0,た/0}、
		\pitch{あ/0,か/1,け/0,れ/0,ば/0}\\(赤い)
		\item \pitch{か/0,る/1,い/1}: 
		\pitch{か/0,る/1,く/1}、
		\pitch{か/0,る/1,さ/1}、
		\pitch{か/0,る/1,く/0,て/0}、
		\pitch{か/0,る/1,か/0,っ/0,た/0}、
		\pitch{か/0,る/1,け/0,れ/0,ば/0}\\(軽い)
	\end{itemize}
	
	\section{Naglasak kod negativnih oblika imenica i pridjeva}
	Svi negativni oblici u japanskom, uključujući oblike imenica, pridjeva i glagola, nezaobilazno se tvore pomoću idjeva \pitch{な/1,い/0}(無い), koji je istovremeno još jedan od \textit{atamadaka} idjeva i znači ``nepostojeći''.
	Naglasci njegovih izvedenih oblika su također \textit{atamadaka}: \pitch{な/1,く/0}、\pitch{な/1,さ/0}、 \pitch{な/1,か/0,っ/0,た/0}、\pitch{な/1,く/0,て/0}、\pitch{な/1,け/0,れ/0,ば/0}.
	
	Kod negativnih oblika imenica i nadjeva, potrebna je još i kopula で + čestica は, ili njihov skraćeni oblik じ\m{ゃ}. One se stapaju s imenicom (odnosno nadjevom) u jednu naglasnu cjelinu prema pravilu \textit{gaženja}.
	Pomoćni idjev ない čini zasebnu naglasnu cjelinu.
	
	Primjeri:
	\begin{itemize}
		\item \textit{heiban} imenica, \pitch{が/0,く/1,せ/1,い/1}(学生):
		%		\pitch{が/0,く/1,せ/1,い/1,じゃ/1,な/1,い/0}(学生じゃない)
		%		\pitch{が/0,く/1,せ/1,い/1,で/1,は/0,な/1,い/0}(学生ではない)
		%		\pitch{が/0,く/1,せ/1,い/1,じゃ/1,な/1,か/0,っ/0,た/0}(学生じゃなかった)
		%		\pitch{が/0,く/1,せ/1,い/1,で/1,は/0,な/1,か/0,っ/0,た/0,/0}(学生ではなかった)
		\item \textit{nakadaka} imenica, \pitch{せ/0,ん/1,せ/1,い/0}(先生):
		\begin{itemize}
			\item \pitch{せ/0,ん/1,せ/1,い/0}+\pitch{じ\m{ゃ}/0}+\pitch{な/1,い/0}=\pitch{せ/0,ん/1,せ/1,い/0,じ\m{ゃ}/0}\!\pitch{な/1,い/0}(先生じゃない)
			\item \pitch{せ/0,ん/1,せ/1,い/0}+\pitch{で/1,は/0}+\pitch{な/1,い/0}=\pitch{せ/0,ん/1,せ/1,い/0,で/0,は/0}\!\pitch{な/1,い/0}(先生ではない)
			\item \pitch{せ/0,ん/1,せ/1,い/0}+\pitch{じ\m{ゃ}/0}+\pitch{な/1,か/0,っ/0,た/0}=\pitch{せ/0,ん/1,せ/1,い/0,じ\m{ゃ}/0}\!\pitch{な/1,か/0,っ/0,た/0}\\(先生じ\m{ゃ}なかった)
			\item \pitch{せ/0,ん/1,せ/1,い/0}+\pitch{で/1,は/0}+\pitch{な/1,か/0,っ/0,た/0}=\pitch{せ/0,ん/1,せ/1,い/0,で/0,は/0}\pitch{な/1,か/0,っ/0,た/0}\\(先生ではなかった)
		\end{itemize}
		\item \textit{heiban} nadjev,
		\item \textit{neki naglašeni} nadjev,
	\end{itemize}
	
	Kod negativnih oblika idjeva, pomoćni idjev ない stapa se sa naglasnom cjelinom osnovne riječi i pritom vrijedi pravilo \textit{gaženja}.
	Primjeri:
	\begin{itemize}
		\item \textit{heiban} idjev,
		\item \textit{neki naglašeni} idjev
	\end{itemize}
		
	
	
	\subsection*{Coming up next...}	
		
	\section{Naglasak kod glagola}
\end{document}