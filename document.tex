\documentclass[12pt]{extarticle}
\usepackage[CJKspace=true]{xeCJK}
\usepackage{fontspec}

\usepackage{booktabs}
\usepackage{tabularx}
\usepackage{ruby}
\usepackage{tipa}
\usepackage{tipx}
\usepackage{tikz}

\usepackage[croatian]{babel}
\usepackage{indentfirst}
\usepackage[a4paper, margin=2.5cm]{geometry}
\usepackage{enumitem}
\usepackage[all]{nowidow}
\usepackage{setspace}
\usepackage{microtype}
\usepackage{multicol}

\setlist[itemize]{itemsep=0em, topsep=0em}
\setlist[enumerate]{itemsep=0em, topsep=0em}

\setCJKmainfont{Noto Serif CJK JP}
\setCJKsansfont{Noto Sans CJK JP}
\setmainfont[Numbers={Lining, Monospaced}, Scale=1.1]{Brill}
\renewcommand{\rubysep}{0ex}
\renewcommand{\rubysize}{0.5}

\newcommand{\pitch}[1]{\raisebox{-\dp\strutbox}{
	\begin{tikzpicture}[
		every node/.style={inner sep=0.0, outer sep=0.0},
		dot/.style={draw, fill, circle, minimum size=0.7ex}
	]
	\foreach[count=\i from 1] \x/\p in {#1} 
		\draw node (P\i) at (\i * 1.1em, 0) {\x\strut} node[dot, anchor=base] (C\i) at (\i * 1.1em, \p * 0.5em + 0.8em) {};
	
	\foreach[count=\i from 1, evaluate=\i as \j using int(\i - 1)] \x/\p in {#1}
		\ifnum \j>0
			\draw ({C\i}.center) -- ({C\j}.center)
		\fi;
	\end{tikzpicture}}
}

\newcommand{\e}{\textbf}
\newcommand{\m}[1]{\kern -0.3em{#1}}

\title{Uvod u japanski naglasni sustav}
\author{Lovriša}

\begin{document}
	\onehalfspacing
	\maketitle
	\pagenumbering{gobble}
	\thispagestyle{empty}
	\tableofcontents
	\newpage
	\pagenumbering{arabic}
	
	\tikzset{
		mytext/.style={text height=1em},
		dot/.style={draw, fill, circle, minimum size=0.7ex, inner sep=0pt, outer sep=0pt}
	}
	
	\section{Uvod}
	Cilj ovog kratkog rada je upoznati čitatelje s osnovama i zainteresirati za daljnje proučavanje naglasnog sustava standardnog tokijskog dijalekta japanskog jezika.
	Japanski naglasni sustav je dosta različit od hrvatskog ili engleskog, stoga je bitno najprije upoznati se s temeljima tog sustava definiranim u ovom odjeljku.
	U idućem odjeljku dan je sustavni pregled tipova naglasaka kroz primjere u kojima se pojavljuju, uz poneko pravilo i najčešće iznimke koje se pojavljuju u svakodnevnom govoru.
	Ostali odjeljci obrađuju pravila vezana uz naglaske kod imenica, pridjeva i glagola.
	
	Pa, što bi Japanci rekli, よろしくおねがいします!
	
	\subsection{M\=ora}
	M\=ora je japanski ekvivalent sloga.
	To je ujedno i najmanja nedjeljiva izgovorna jedinica neke riječi.
	Karakteristično obilježje japanskog jezika jest da svaka m\=ora ima jednako trajanje.
	U načelu, 1 m\=ora je reprezentirana 1 hiraganom, osim malih ゃ, ゅ, ょ koji su nalijepljeni na drugu hiraganu, npr. u しゃ, しゅ, しょ.
	,,Samoglasno N'' ん se broji kao 1 m\=ora.
	I \textit{sokuon} っ (npr. u げっかん) i \textit{ch\=oonpo} ー  (npr. u ハート) se broje kao 1 m\=ora.
	U tablici \ref{tab:mora-type} ispod kopiranoj s Wikipedije navedene su vrste m\=ora za radoznale:
	
	\begin{table}[htb]
		\centering
		\caption{Vrste m\=ora.}
		\label{tab:mora-type}
		\begin{tabular}{llll}
			\toprule
			& & & Br. m\=ora \\
			\multicolumn{2}{l}{Vrsta} & Primjer & (u primjeru) \\
			\midrule
			V & samoglasnici & \e{お} & 1 \\
			jV & palatalizirani samoglasnici & \e{よ} (世) & 1 \\
			CV & suglasnik i samoglasnik & \e{こ} (子) & 1 \\
			CjV & suglasnik i pal. samoglasnik & \e{きょ} (巨) & 1 \\
			R & produljenje samoglasnika & きょ\e{う} (今日)、キ\e{ー}  & 2 \\
			N & samoglasno N & こ\e{ん} (紺) & 2 \\
			Q & udvostručavanje suglasnika & こ\e{っ}こ (国庫) & 3 \\
			\bottomrule
		\end{tabular}
	\end{table}

	Određena riječ može imati naglasak na samo jednoj, ili nijednoj m\=ori.
	Od navedenih vrsta, samo V, jV, CV i CjV mogu nositi naglasak.
	
	\subsection{Tonovi, intonacija i naglasak}
	U japanskom tehnički razlikujemo samo dvije vrste tona: visoki i niski.
	Svaka m\=ora ima točno jedan od ta dva tona, ton se ne može mijenjati unutar iste m\=ore.
	Pri promjeni iz niskog tona u visoki ili obratno se osjeti razlika u visini; međutim, i među tonovima iste vrste visina također može blago varirati.
	Prirodno u izgovoru visina opada prema kraju naglasne cjeline.
	
	Intonacija je kretanje visine tonova u izgovorenoj riječi.
	Intonacija za sve riječi koje nemaju naglasak je ista --- neutralna, a za riječi koje naglasak imaju ovisi o mjestu gdje se naglasak nalazi.
	U japanskom jeziku neutralna intonacija je pomalo neobična: počinje s niskim tonom na prvoj m\=ori, a sve m\=ore poslije prve imaju visoki ton.
	Primjeri:
	\begin{multicols}{3}
		\begin{itemize}
			\item \pitch{に/0,ほ/1,ん/1,ご/1}(日本語)
			\item \pitch{こ/0,ん/1,に/1,ち/1,は/1}
			\item \pitch{く/0,る/1,ま/1}(車)
			\item \pitch{り/0,ん/1,ご/1}
			\item \pitch{し/0,ん/1,ぶ/1,ん/1}(新聞)
			\item \pitch{と/0,も/1,だ/1,ち/1}(友達)
		\end{itemize}
	\end{multicols}

	U japanskom jeziku naglasak se manifestira tako što nakon m\=ore koja je naglašena visina intonacije naglo padne, odnosno naglašena m\=ora je posljednja visoka u riječi.
	Primjeri (naglašena m\=ora je podebljana):
	\begin{multicols}{3}
		\begin{itemize}
			\item \pitch{\e{ね}/1,こ/0} (猫)
			\item \pitch{な/0,\e{が}/1,さ/0,き/0}(長崎)
			\item \pitch{に/0,ほ/1,ん/1,\e{じ}/1,ん/0}\\(日本人)
			\item \pitch{い/0,も/1,う/1,\e{と}/1}(妹)
			\item \pitch{コ/0,ー/1,\e{ヒ}/1,ー/0}
			\item \pitch{ピ/0,\e{カ}/1,チ\m{ュ}/0,ウ/0}
		\end{itemize}
	\end{multicols}
	
	\subsection{Naglasna cjelina}
	Naglasnu cjelinu čini niz riječi koji zajedno ima najviše jedan naglasak.
	Svaka riječ inherentno ima ili nema određeni naglasak, no kad se više riječi udruži u naglasnu cjelinu, naglasak u pojedinim riječima može se izgubiti ili pomaknuti prema određenim pravilima.
	Najčešće se radi o glavnoj riječi sa zanaglasnicama (čestice, kopula, pomoćni glagoli), ili o konjugiranom obliku neke riječi.
	Neki primjeri naglasnih cjelina (naglašena m\=ora, ako postoji, je podebljana):
	\begin{itemize}
		\item \pitch{こ/0,れ/1}+\pitch{は/0}=\pitch{こ/0,れ/1,は/1}
		\item \pitch{た/0,な/1,か/1}+\pitch{\e{で}/1,す/0}=\pitch{た/0,な/1,か/1,\e{で}/1,す/0}(田中です)
		\item \pitch{お/0,と/1,\e{こ}/1}+\pitch{\e{で}/1,す/0}=\pitch{お/0,と/1,\e{こ}/1,で/0,す/0}(男です)
		\item \pitch{\e{じ}/1,し\m{ょ}/0}+\pitch{\e{で}/1,す/0}=\pitch{\e{じ}/1,し\m{ょ}/0,で/0,す/0}(辞書です)
		\item \pitch{た/0,\e{べ}/1,る/0}+\pitch{さ/0,せ/1,る/1}+\pitch{ら/0,れ/1,る/1}
		=\pitch{た/0,べ/1,さ/1,せ/1,ら/1,\e{れ}/1,る/0}(食べさせられる)
		\item \pitch{あ/0,た/1,た/1,\e{か}/1,い/0}(暖かい)、\pitch{あ/0,た/1,\e{た}/1,か/0,く/0,て/0}(暖かくて)
	\end{itemize}
	
	\section{Naglasni tipovi}
	S obzirom na poziciju unutar riječi, naglaske možemo podijeliti u četiri kategorije:
	\begin{enumerate}
		\item \textit{heiban} 平板 --- neutralna intonacija, nema naglaska
		\item \textit{atamadaka} 頭高 --- naglasak je na prvoj m\=ori (glavi)
		\item \textit{nakadaka} 中高 --- naglasak nije na prvoj ni na zadnjoj m\=ori
		\item \textit{odaka} 尾高 --- naglasak je na zadnjoj m\=ori (repu)
	\end{enumerate}
	Iako možemo reći da se u stvari radi o samo \textit{dvije} kategorije (bez naglaska -- \textit{heiban}, i s naglaskom -- sve ostale), svaka od navedenih kategorija ima svoje posebnosti s obzirom na promjene riječi (konjugacije), stoga je te promjene lakše promatrati kroz 4 kategorije.
	
	Naglasci se u rječnicima još označavaju brojem prema m\=ori koja je naglašena.
	Tako se \textit{atamadaka} označava sa (1), \textit{nakadaka} brojevima od (2) nadalje, a \textit{odaka} brojem zadnje m\=ore u riječi ili, prema konvenciji negativnog indeksiranja\footnote{ovu konvenciju popularizirao je programski jezik Python, a radi se o brojanju elemenata niza unatrag, (-1) predstavlja zadnji, (-2) predzadnji element, itd.}, sa (-1).
	Heiban se označava sa (0).
	
	Kod samostalnih riječi, \textit{heiban} i \textit{odaka} se ne razlikuju, jer intonacija nikad ne padne ni u jednom ni u drugom slučaju.
	Međutim, kada te riječi čine naglasnu cijelinu s drugim riječima, razlika u intonaciji postane primjetljiva:
	\begin{itemize}
		\item \pitch{は/0,な/1}(鼻)、\pitch{は/0,な/1,が/1}(鼻が)--- \textit{heiban}
		\item \pitch{は/0,な/1}(花)、\pitch{は/0,な/1,が/0}(花が)--- \textit{odaka}
	\end{itemize}

	Također, možemo primijetiti kako su visine tonova prve dvije m\=ore unutar neke naglasne cjeline međusobno uvijek različite (bilo niska-visoka, ili visoka-niska), pa se po tome lako može prepoznati početak riječi ili naglasne cjeline.
	
	\subsection{Naglasni tipovi s obzirom na broj m\=ora u riječi}
	\subsubsection*{1 m\=ora}
	Riječi koje se sastoje od 1 m\=ore koje počinju niskom morom imaju \textit{heiban} tip naglaska, a koje počinju visokom \textit{atamadaka} vrstu naglaska.
	Drugi tipovi nisu mogući za riječi s 1 m\=orom.
	Primjeri riječi različitih naglasaka: (skupa s česticom)
	\begin{enumerate}
		\item \textit{heiban}
		\begin{multicols}{3}
			\begin{itemize}
				\item \pitch{き/0,が/1}(気が)
				\item \pitch{ひ/0,が/1}(日が)
				\item \pitch{は/0,が/1}(葉が)
			\end{itemize}
		\end{multicols}
		\item \textit{atamadaka}
		\begin{multicols}{3}
			\begin{itemize}
				\item \pitch{き/1,が/0}(木が)
				\item \pitch{ひ/1,が/0}(火が)
				\item \pitch{は/1,が/0}(歯が)
			\end{itemize}
		\end{multicols}
	\end{enumerate}
	
	\subsubsection*{2 m\=ore}
	Riječi koje se sastoje od 2 m\=ore mogu imati sve tipove osim \textit{nakadaka}.
	Primjeri:
	\begin{enumerate}
		\item \textit{heiban}
		\begin{multicols}{3}
			\begin{itemize}
				\item \pitch{く/0,ち/1,が/1}(口)
				\item \pitch{は/0,な/1,が/1}(鼻)
				\item \pitch{く/0,び/1,が/1}(首)
			\end{itemize}
		\end{multicols}
		\item \textit{atamadaka}
		\begin{multicols}{3}
			\begin{itemize}
				\item \pitch{か/1,た/0,が/0}(肩)
				\item \pitch{の/1,ど/0,が/0}(喉)
			\end{itemize}
		\end{multicols}
		\item \textit{odaka}
		\begin{multicols}{3}
			\begin{itemize}
				\item \pitch{か/0,み/1,が/0}(髪)
				\item \pitch{み/0,み/1,が/0}(耳)
				\item \pitch{む/0,ね/1,が/0}(胸)
				\item \pitch{は/0,ら/1,が/0}(腹)
				\item \pitch{ゆ/0,び/1,が/0}(指)
				\item \pitch{あ/0,し/1,が/0}(足)
			\end{itemize}
		\end{multicols}
	\end{enumerate}

	\subsubsection*{3 m\=ore}
	Tek riječi od 3 i više m\=ora mogu imati sve tipove naglasaka.
	One su najčešće \textit{heiban} i \textit{atamadaka}, a rjeđe \textit{nakadaka} ili \textit{odaka}, stoga je korisno zapamtiti česte riječi koje spadaju pod zadnje dvije kategorije.
	Primjeri:
	\begin{enumerate}
		\item \textit{heiban}
		\begin{multicols}{3}
			\begin{itemize}
				\item \pitch{ふ/0,し/1,ぎ/1,な/1}
				\item \pitch{カ/0,バ/1,ン/1,は/1}
				\item \pitch{か/0,ん/1,じ/1,は/1}(漢字は)
				\item \pitch{つ/0,ば/1,さ/1,は/1}(翼は)
				\item \pitch{う/0,わ/1,さ/1,は/1}(噂は)
				\item \pitch{し/0,か/1,た/1,は/1}(仕方は)
			\end{itemize}
		\end{multicols}
		\item \textit{atamadaka}
		\begin{multicols}{3}
			\begin{itemize}
				\item \pitch{ご/1,は/0,ん/0,は/0}(ご飯は)
				\item \pitch{ご/1,ぜ/0,ん/0,は/0}(午前は)
				\item \pitch{か/1,ぞ/0,く/0,は/0}(家族は)
				\item \pitch{じ/1,ん/0,じ\m{ゃ}/0,は/0}(神社は)
				\item \pitch{め/1,が/0,ね/0,は/0}
				\item \pitch{け/1,し/0,き/0,は/0}(景色は)
				\item \pitch{に/1,も/0,つ/0,は/0}(荷物は)
				\item \pitch{で/1,ん/0,き/0,は/0}(電気は)
				\item \pitch{て/1,ん/0,き/0,は/0}(天気は)
				\item \pitch{げ/1,ん/0,き/0,な/0}(元気な)
				\item \pitch{き/1,れ/0,い/0,な/0}
			\end{itemize}
		\end{multicols} \pagebreak %dirty
		\item \textit{nakadaka}
		\begin{multicols}{3}
			\begin{itemize}
				\item \pitch{す/0,こ/1,し/0,は/0}(少しは)
				\item \pitch{ひ/0,と/1,り/0,は/0}(一人は)
				\item \pitch{ひ/0,と/1,つ/0,は/0}(一つは)
				\item \pitch{い/0,つ/1,つ/0,は/0}(五つは)
				\item \pitch{き/0,の/1,う/0,は/0}(昨日は)
				\item \pitch{に/0,お/1,い/0,は/0}(匂いは)
				\item \pitch{あ/0,な/1,た/0,は/0} 
			\end{itemize}
		\end{multicols}
		\item \textit{odaka}
		\begin{multicols}{3}
			\begin{itemize}
				\item \pitch{あ/0,し/1,た/1,は/0}(明日は)
				\item \pitch{こ/0,と/1,ば/1,は/0}(言葉は)
				\item \pitch{ふ/0,た/1,り/1,は/0}(二人は)
				\item \pitch{ふ/0,た/1,つ/1,は/0}(二つは)
				\item \pitch{み/0,っ/1,つ/1,は/0}(三つは)
				\item \pitch{よ/0,っ/1,つ/1,は/0}(四つは)
				\item \pitch{み/0,ん/1,な/1,は/0}(皆は)
				\item \pitch{お/0,と/1,こ/1,は/0}(男は)
				\item \pitch{お/0,ん/1,な/1,は/0}(女は)
				\item \pitch{あ/0,た/1,ま/1,は/0}(頭は)
			\end{itemize}
		\end{multicols}
	\end{enumerate}
	
	\subsubsection*{4 m\=ore}
	Često su imenice od 4 m\=ore tipa \textit{heiban}, pogotovo one koje se pišu s dva kanjija.
	Stoga opet valja zapamtiti česte iznimke od ovoga pravila.
	Primjeri: (ovaj put popratne čestice su izostavljene radi bolje preglednosti)
		\begin{enumerate}
		\item \textit{heiban}
		\begin{multicols}{3}
			\begin{itemize}
				\item \pitch{が/0,っ/1,こ/1,う/1}(学校)
				\item \pitch{ぎ/0,ん/1,こ/1,う/1}(銀行)
				\item \pitch{さ/0,い/1,て/1,い/1}(最低)
				\item \pitch{さ/0,い/1,あ/1,く/1}(最悪)
				\item \pitch{さ/0,い/1,こ/1,う/1}(最高)
				\item \pitch{へ/0,い/1,ば/1,ん/1}(平板)
				\item \pitch{べ/0,ん/1,き\m{ょ}/1,う/1}(勉強)
				\item \pitch{し/0,ん/1,ぱ/1,い/1}(心配)
			\end{itemize}
		\end{multicols}
		\item \textit{atamadaka}
		\begin{multicols}{3}
			\begin{itemize}
				\item \pitch{お/1,ん/0,が/0,く/0}(音楽)
				\item \pitch{あ/1,い/0,さ/0,つ/0}(挨拶)
				\item \pitch{め/1,い/0,わ/0,く/0}(迷惑)
				\item \pitch{ま/1,い/0,に/0,ち/0}(毎日)
				\item \pitch{ま/1,い/0,あ/0,さ/0}(毎朝)
				\item \pitch{ま/1,い/0,ば/0,ん/0}(毎晩)
			\end{itemize}
		\end{multicols} \pagebreak %dirty
		\item \textit{nakadaka}
		\begin{multicols}{3}
			\begin{itemize}
				\item \pitch{せ/0,ん/1,せ/1,い/0}(先生)
				\item \pitch{は/0,ん/1,ぶ/1,ん/0}(半分)
				\item \pitch{あ/0,ん/1,な/1,い/0}(案内)
				\item \pitch{お/0,と/1,と/1,い/0}(一昨日)
				\item \pitch{ア/0,パ/1,ー/0,ト/0}
				\item \pitch{し/0,つ/1,れ/0,い/0}(失礼)
				\item \pitch{の/0,み/1,も/0,の/0}(飲み物)
			\end{itemize}
		\end{multicols}
		\item \textit{odaka}
		\begin{multicols}{2}
			\begin{itemize}
				\item \pitch{い/0,も/1,う/1,と/1}(妹)
				\item \pitch{お/0,と/1,う/1,と/1}(弟)
			\end{itemize}
		\end{multicols}
	\end{enumerate}
	
	\subsection{Još par korisnih pravila}
	Dva su korisna pravila vezana uz zamjenice:
	\begin{enumerate}
		\item Zamjenice koje počinju na k-, s-, a- su \textit{heiban}:
			\begin{multicols}{3}
				\begin{itemize} 
					\item \pitch{こ/0,れ/1}
					\item \pitch{そ/0,こ/1}
					\item \pitch{あ/0,の/1}
					\item \pitch{そ/0,う/1}
					\item \pitch{こ/0,ち/1,ら/1}
					\item itd.
				\end{itemize}
			\end{multicols}
		\item Upitne zamjenice su \textit{atamadaka}:
			\begin{multicols}{5}
				\begin{itemize}
					\item \pitch{ど/1,れ/0}
					\item \pitch{ど/1,こ/0}
					\item \pitch{ど/1,の/0}
					\item \pitch{ど/1,う/0}
					\item \pitch{ど/1,ち/0,ら/0}
					\item \pitch{だ/1,れ/0}
					\item \pitch{い/1,つ/0}
					\item \pitch{な/1,に/0}
					\item \pitch{な/1,ぜ/0}
					\item itd.
				\end{itemize}
			\end{multicols}
	\end{enumerate}

	Iznimno, uz uobičajenu verziju zamjenice そう postoji još jedna koja se više upotrebljava kao imenica, a ona je \textit{atamadaka}.
	Prva je priložna, pa ona opisuje glagole:
	\begin{itemize}
		\begin{multicols}{3}
			\item \pitch{そ/0,う/1,い/1,う/1}
			\item \pitch{そ/0,う/1,す/1,る/1}
			\item \pitch{そ/0,う/1,し/1,よ/1,う/1}
		\end{multicols}
	\end{itemize}
	Druga se koristi kao imenica (sufiksna imenica), nju prepoznajemo po tome što iza nje može doći kopula:
	\begin{itemize}
		\begin{multicols}{3}
			\item \pitch{そ/1,う/0,で/0,す/0,か/0}?
			\item \pitch{そ/1,う/0,だ/0,な/0}…
			\item \pitch{ふ/0,り/1,そ/1,う/0,で/0,す/0}\\(降りそうです\footnote{više o promjeni naglasaka kod glagola kasnije})
		\end{multicols}
	\end{itemize}
	

	\section{Naglasak kod imenica}
	Imenice mogu imati bilo koji od 4 naglaska, kao što je pokazano na nizu primjera u prethodnom odjeljku.
	Specijalni slučaj čine dugačke imenice koje su nastale srastanjem dviju ili više drugih imenica.
	Njihov tip naglaska je gotovo uvijek \textit{nakadaka}, i to takav da je naglasak na prvoj mori druge (ili posljednje) riječi u nizu.
	Pritom nije bitno kakav naglasak imaju individualne riječi.
	
	Primjeri: (na pojedinačne riječi su dodane čestice kako bi se primijetila razlika između \textit{heiban} i \textit{odaka} naglasaka)
	\begin{itemize}
		\item \pitch{や/0,ま/1,が/0}(山が)+\pitch{の/0,ぼ/1,り/1,が/1}(登りが)=\pitch{や/0,ま/1,の/1,ぼ/0,り/0}(山登り)
		\item \pitch{は/0,く/1,じ/1,が/1}(白人が)+\pitch{だ/0,ん/1,せ/1,い/1,が/1}(男性が)=%
		\begin{minipage}{3cm}\pitch{は/0,く/1,じ/1,ん/1,だ/0,ん/0,せ/0,い/0}\\(白人男性)\end{minipage}
	\end{itemize}
	
	Međutim, ima i riječi koje ne slijede gore navedeno pravilo o naglasku na početku druge riječi.
	Često korišten primjer je riječi za jezik i nacionalnost, koje unose neviđen kuršlus u naglaske:
	\begin{itemize}
		\item \pitch{に/0,ほ/1,ん/0}(日本)、\pitch{に/0,ほ/1,ん/1,ご/1}(日本語)、\pitch{に/0,ほ/1,ん/1,じ/1,ん/0}(日本人)
		\item \pitch{ち\m{ゅ}/1,う/0,ご/0,く/0}(中国)、\pitch{ち\m{ゅ}/0,う/1,ご/1,く/1,ご/1}、\pitch{ち\m{ゅ}/0,う/1,ご/1,く/1,じ/0,ん/0}
		\item \pitch{か/1,ん/0,こ/0,く/0}(韓国)、\pitch{か/0,ん/1,こ/1,く/1,ご/1}、\pitch{か/0,ん/1,こ/1,く/1,じ/0,ん/0}
		\item \pitch{ど/1,い/0,つ/0}(ドイツ)、\pitch{ど/0,い/1,つ/1,ご/1}、\pitch{ど/0,い/1,つ/1,じ/0,ん/0}
		\item \pitch{ふ/0,ら/1,ん/1,す/1}(フランス)、\pitch{ふ/0,ら/1,ん/1,す/1,ご/1}、\pitch{ふ/0,ら/1,ん/1,す/1,じ/0,ん/0}
		\item \pitch{く/0,ろ/1,あ/1,ち/1,あ/1}(クロアチア)、\pitch{く/0,ろ/1,あ/1,ち/1,あ/1,ご/0}、\pitch{く/0,ろ/1,あ/1,ち/1,あ/1,じ/0,ん/0}
		\item \pitch{せ/0,る/1,び/1,あ/1}(セルビア)、\pitch{せ/0,る/1,び/1,あ/1,ご/1}、\pitch{せ/0,る/1,び/1,あ/1,じ/0,ん/0}
		\item \pitch{ぼ/0,す/1,に/1,あ/1}(ボスニア)、\pitch{ぼ/0,す/1,に/1,あ/1,ご/0}、\pitch{ぼ/0,す/1,に/1,あ/1,じ/0,ん/0}
	\end{itemize}
	
	\subsection{Naglasak u čestim kombinacijama imenica s drugim riječima}
	
	\section*{Coming up next...}
	\section{Naglasak kod pridjeva}
	\subsection{Naglasak kod osnovnih oblika idjeva}
	\subsection{Naglasak kod osnovnih oblika nadjeva}
	\subsection{Naglasak kod negativnih oblika pridjeva}
	
	\section{Naglasak kod glagola}
\end{document}