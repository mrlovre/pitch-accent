% !TeX root = document.tex
S obzirom na poziciju unutar riječi, naglaske možemo podijeliti u četiri kategorije:
\begin{enumerate}
	\item \textit{heiban} 平板 --- neutralna intonacija, nema naglaska
	\item \textit{atamadaka} 頭高 --- naglasak je na prvoj m\=ori (glavi)
	\item \textit{nakadaka} 中高 --- naglasak nije na prvoj ni na zadnjoj m\=ori
	\item \textit{odaka} 尾高 --- naglasak je na zadnjoj m\=ori (repu)
\end{enumerate}
Iako možemo reći da se u stvari radi o samo \textit{dvije} kategorije (bez naglaska --- \textit{heiban}, i s naglaskom --- sve ostale), svaka od navedenih kategorija ima svoje posebnosti s obzirom na promjene riječi (konjugacije), stoga je te promjene lakše promatrati kroz 4 kategorije.

Naglasci se u rječnicima još označavaju brojem prema m\=ori koja je naglašena.
Tako se \textit{atamadaka} označava sa [1], \textit{nakadaka} brojevima od [2] nadalje, a \textit{odaka} brojem zadnje m\=ore u riječi ili, prema konvenciji negativnog indeksiranja\footnote{ovu konvenciju popularizirao je programski jezik Python, a radi se o brojanju elemenata niza unatrag, [-1] predstavlja zadnji, [-2] predzadnji element, itd.}, sa [-1].
Heiban se označava sa [0].

Kod samostalnih riječi, \textit{heiban} i \textit{odaka} se ne razlikuju, jer intonacija nikad ne padne ni u jednom ni u drugom slučaju.
Međutim, kada te riječi čine naglasnu cijelinu s drugim riječima, razlika u intonaciji postane primjetljiva:
\begin{itemize}
	\item \pitch{は/0,な/1}(鼻)、\pitch{は/0,な/1,\e{が}/1}(鼻が)--- \textit{heiban}
	\item \pitch{は/0,な/1}(花)、\pitch{は/0,な/1,\e{が}/0}(花が)--- \textit{odaka}
\end{itemize}

Također, možemo primijetiti kako su visine tonova prve dvije m\=ore unutar neke naglasne cjeline međusobno uvijek različite (bilo niska-visoka, ili visoka-niska), pa se po tome lako može prepoznati početak riječi ili naglasne cjeline.

\subsection{Naglasni tipovi s obzirom na broj m\=ora u riječi}
\subsubsection*{1 m\=ora}
Riječi koje se sastoje od 1 m\=ore koje počinju niskom morom su \textit{heiban}, a one koje počinju visokom \textit{atamadaka}.
Drugi naglasni tipovi nisu mogući za riječi s 1 m\=orom.
Primjeri riječi različitih naglasnih tipova (skupa s česticom radi lakšeg razlikovanja):
\begin{enumerate}
	\item \textit{heiban}
	\begin{multicols}{3}
		\begin{itemize}
			\item \pitch{き/0,が/1}(気が)
			\item \pitch{ひ/0,が/1}(日が)
			\item \pitch{は/0,が/1}(葉が)
		\end{itemize}
	\end{multicols}
	\item \textit{atamadaka}
	\begin{multicols}{3}
		\begin{itemize}
			\item \pitch{き/1,が/0}(木が)
			\item \pitch{ひ/1,が/0}(火が)
			\item \pitch{は/1,が/0}(歯が)
		\end{itemize}
	\end{multicols}
\end{enumerate}

\subsubsection*{2 m\=ore}
Riječi koje se sastoje od 2 m\=ore mogu imati sve tipove osim \textit{nakadaka}.
Primjeri:
\begin{enumerate}
	\item \textit{heiban}
	\begin{multicols}{3}
		\begin{itemize}
			\item \pitch{く/0,ち/1,が/1}(口)
			\item \pitch{は/0,な/1,が/1}(鼻)
			\item \pitch{く/0,び/1,が/1}(首)
		\end{itemize}
	\end{multicols}
	\item \textit{atamadaka}
	\begin{multicols}{3}
		\begin{itemize}
			\item \pitch{か/1,た/0,が/0}(肩)
			\item \pitch{の/1,ど/0,が/0}(喉)
		\end{itemize}
	\end{multicols}
	\item \textit{odaka}
	\begin{multicols}{3}
		\begin{itemize}
			\item \pitch{か/0,み/1,が/0}(髪)
			\item \pitch{み/0,み/1,が/0}(耳)
			\item \pitch{む/0,ね/1,が/0}(胸)
			\item \pitch{は/0,ら/1,が/0}(腹)
			\item \pitch{ゆ/0,び/1,が/0}(指)
			\item \pitch{あ/0,し/1,が/0}(足)
		\end{itemize}
	\end{multicols}
\end{enumerate}

\subsubsection*{3 m\=ore}
Tek riječi od 3 i više m\=ora mogu imati sve tipove naglasaka.
One su najčešće \textit{heiban} i \textit{atamadaka}, a rjeđe \textit{nakadaka} ili \textit{odaka}, stoga je korisno zapamtiti česte riječi koje spadaju pod zadnje dvije kategorije.
Primjeri:
\begin{enumerate}
	\item \textit{heiban}
	\begin{multicols}{3}
		\begin{itemize}
			\item \pitch{ふ/0,し/1,ぎ/1,な/1}
			\item \pitch{カ/0,バ/1,ン/1,は/1}
			\item \pitch{か/0,ん/1,じ/1,は/1}(漢字は)
			\item \pitch{つ/0,ば/1,さ/1,は/1}(翼は)
			\item \pitch{う/0,わ/1,さ/1,は/1}(噂は)
			\item \pitch{し/0,か/1,た/1,は/1}(仕方は)
		\end{itemize}
	\end{multicols}
	\item \textit{atamadaka}
	\begin{multicols}{3}
		\begin{itemize}
			\item \pitch{ご/1,は/0,ん/0,は/0}(ご飯は)
			\item \pitch{ご/1,ぜ/0,ん/0,は/0}(午前は)
			\item \pitch{か/1,ぞ/0,く/0,は/0}(家族は)
			\item \pitch{じ/1,ん/0,じ\m{ゃ}/0,は/0}(神社は)
			\item \pitch{め/1,が/0,ね/0,は/0}
			\item \pitch{け/1,し/0,き/0,は/0}(景色は)
			\item \pitch{に/1,も/0,つ/0,は/0}(荷物は)
			\item \pitch{で/1,ん/0,き/0,は/0}(電気は)
			\item \pitch{て/1,ん/0,き/0,は/0}(天気は)
			\item \pitch{げ/1,ん/0,き/0,な/0}(元気な)
			\item \pitch{き/1,れ/0,い/0,な/0}
		\end{itemize}
	\end{multicols} %\pagebreak %dirty
	\item \textit{nakadaka}
	\begin{multicols}{3}
		\begin{itemize}
			\item \pitch{す/0,こ/1,し/0,は/0}(少しは)
			\item \pitch{ひ/0,と/1,り/0,は/0}(一人は)
			\item \pitch{ひ/0,と/1,つ/0,は/0}(一つは)
			\item \pitch{い/0,つ/1,つ/0,は/0}(五つは)
			\item \pitch{き/0,の/1,う/0,は/0}(昨日は)
			\item \pitch{に/0,お/1,い/0,は/0}(匂いは)
			\item \pitch{あ/0,な/1,た/0,は/0} 
		\end{itemize}
	\end{multicols}
	\item \textit{odaka}
	\begin{multicols}{3}
		\begin{itemize}
			\item \pitch{あ/0,し/1,た/1,は/0}(明日は)
			\item \pitch{こ/0,と/1,ば/1,は/0}(言葉は)
			\item \pitch{ふ/0,た/1,り/1,は/0}(二人は)
			\item \pitch{ふ/0,た/1,つ/1,は/0}(二つは)
			\item \pitch{み/0,っ/1,つ/1,は/0}(三つは)
			\item \pitch{よ/0,っ/1,つ/1,は/0}(四つは)
			\item \pitch{み/0,ん/1,な/1,は/0}(皆は)
			\item \pitch{お/0,と/1,こ/1,は/0}(男は)
			\item \pitch{お/0,ん/1,な/1,は/0}(女は)
			\item \pitch{あ/0,た/1,ま/1,は/0}(頭は)
		\end{itemize}
	\end{multicols}
\end{enumerate}

\subsubsection*{4 m\=ore}
Često su imenice od 4 m\=ore \textit{heiban}, pogotovo one koje se pišu s dva kanđija.
Stoga opet valja zapamtiti česte iznimke od ovoga pravila.
Primjeri: (ovaj put popratne čestice su izostavljene radi bolje preglednosti)
\begin{enumerate}
	\item \textit{heiban}
	\begin{multicols}{3}
		\begin{itemize}
			\item \pitch{が/0,っ/1,こ/1,う/1}(学校)
			\item \pitch{ぎ/0,ん/1,こ/1,う/1}(銀行)
			\item \pitch{さ/0,い/1,て/1,い/1}(最低)
			\item \pitch{さ/0,い/1,あ/1,く/1}(最悪)
			\item \pitch{さ/0,い/1,こ/1,う/1}(最高)
			\item \pitch{へ/0,い/1,ば/1,ん/1}(平板)
			\item \pitch{べ/0,ん/1,き\m{ょ}/1,う/1}(勉強)
			\item \pitch{し/0,ん/1,ぱ/1,い/1}(心配)
		\end{itemize}
	\end{multicols}
	\item \textit{atamadaka}
	\begin{multicols}{3}
		\begin{itemize}
			\item \pitch{お/1,ん/0,が/0,く/0}(音楽)
			\item \pitch{あ/1,い/0,さ/0,つ/0}(挨拶)
			\item \pitch{め/1,い/0,わ/0,く/0}(迷惑)
			\item \pitch{ま/1,い/0,に/0,ち/0}(毎日)
			\item \pitch{ま/1,い/0,あ/0,さ/0}(毎朝)
			\item \pitch{ま/1,い/0,ば/0,ん/0}(毎晩)
		\end{itemize}
	\end{multicols} %\pagebreak %dirty
	\item \textit{nakadaka}
	\begin{multicols}{3}
		\begin{itemize}
			\item \pitch{せ/0,ん/1,せ/1,い/0}(先生)
			\item \pitch{は/0,ん/1,ぶ/1,ん/0}(半分)
			\item \pitch{あ/0,ん/1,な/1,い/0}(案内)
			\item \pitch{お/0,と/1,と/1,い/0}(一昨日)
			\item \pitch{ア/0,パ/1,ー/0,ト/0}
			\item \pitch{し/0,つ/1,れ/0,い/0}(失礼)
			\item \pitch{の/0,み/1,も/0,の/0}(飲み物)
		\end{itemize}
	\end{multicols}
	\item \textit{odaka}
	\begin{multicols}{3}
		\begin{itemize}
			\item \pitch{い/0,も/1,う/1,と/1}(妹)
			\item \pitch{お/0,と/1,う/1,と/1}(弟)
		\end{itemize}
	\end{multicols}
\end{enumerate}

\subsection{Još par korisnih pravila}
Dva su korisna pravila vezana uz zamjenice:
\begin{enumerate}
	\item Zamjenice koje počinju na k-, s-, a- su \textit{heiban}:
	\begin{multicols}{6}
		\begin{itemize} 
			\item \pitch{こ/0,れ/1}
			\item \pitch{そ/0,こ/1}
			\item \pitch{あ/0,の/1}
			\item \pitch{そ/0,う/1}
			\item \pitch{こ/0,ち/1,ら/1}
			\item \vphantom{\pitch{か/1}}itd.
		\end{itemize}
	\end{multicols}
	\item Upitne zamjenice su \textit{atamadaka}:
	\begin{multicols}{6}
		\begin{itemize}
			\item \pitch{ど/1,れ/0}
			\item \pitch{ど/1,こ/0}
			\item \pitch{ど/1,の/0}
			\item \pitch{ど/1,う/0}
			\item \pitch{ど/1,ち/0,ら/0}
			\item \pitch{だ/1,れ/0}
			\item \pitch{い/1,つ/0}
			\item \pitch{な/1,に/0}
			\item \pitch{な/1,ぜ/0}
			\item \vphantom{\pitch{か/1}}itd.
		\end{itemize}
	\end{multicols}
\end{enumerate}

Uz zamjenicu そう postoji i sufiks 〜そう, točnije sufiksna imenica, a ona je \textit{atamadaka}.
Nju razlikujemo od zamjenice prema tome što iza nje može doći kopula:
\begin{itemize}
	\begin{multicols}{3}
		\item \pitch{そ/1,う/0,で/0,す/0,か/0}?
		\item \pitch{そ/1,う/0,だ/0,な/0}…
		\item \pbox{\textwidth}{\pitch{ふ/0,り/1,そ/1,う/0,で/0,す/0}\bh(降りそうです)}\footnote{više o naglasku kod glagola u odjeljku \ref{sec:glagoli}}
		
	\end{multicols}
\end{itemize}
S druge strane, zamjenica そう je priložna i ona opisuje glagole:
\begin{itemize}
	\begin{multicols}{3}
		\item \pitch{そ/0,う/1,い/1,う/1}
		\item \pitch{そ/0,う/1,す/1,る/1}
		\item \pitch{そ/0,う/1,し/1,よ/1,う/1}
	\end{multicols}
\end{itemize}
