% !TeX root = document.tex
\label{sec:glagoli}
Glagoli su u japanskom jeziku najpromjenjivija vrsta riječi.

Osvrnemo li se na dosad obrađene vrste riječi, možemo primijetiti da se imenice i nadjevi uopće ne mijenjaju s obzirom na vrijeme, negaciju ili modalitet.
Ono što se kod njih mijenja je kopula, jedini problem s naglaskom u tim slučajevima jest gdje će ga novonastala naglasna cjelina imati.
To se vrlo uspješno dalo objasniti pravilom gaženja.

Sljedeći na red su došli idjevi, koji su, za razliku od prethodne dvije vrste, promjenjivi.
Zavšetak 〜い mijenja se u 〜く ili 〜け, te se dodaju nastavci (npr. 〜くない ili 〜ければ).
Tu se dogodila nova promjena u naglasku u slučaju \textit{nakadaka} idjeva od 3 m\=ore: naglasak se kod izvedenih oblika pomicao s 2. na 1. m\=oru.

Kod glagola se mijenja čak i osnova.
Točnije, kod \textit{godan} glagola imamo 5 osnova --- po jednu za svaki samoglasnik, a kod \textit{ićidan} glagola jednu.
K tome se još na osnove dodaju različiti nastavci, pomoćni glagoli i idjevi, pa je za očekivati da će naglasak dosta divljati uza sve te promjene.
Sada će se klasifikacija naglasaka u 4 tipa koja je uvedena na početku po prvi put zbilja pokazati korisnom: za svaki tip vrijede ista pravila bez obzira bio glagol \textit{ićidan} ili \textit{godan}.

Pa, krenimo s najčešćim glagolima svakog tipa.

\subsection{Klasifikacija najčešćih glagola prema naglasnim tipovima}
Slično kao i idjevi, glagoli također mogu biti \textit{heiban}, \textit{atamadaka} ili \textit{nakadaka}.
\textit{Odaka} glagoli ne postoje.
Isto tako, slično kao i idjevi, svi \textit{nakadaka} glagoli naglasak imaju na [-2] mori.

\subsubsection*{\textit{Heiban} glagoli}
\textit{Heiban} glagoli se često nesvjesno miješaju s \textit{nakadaka} glagolima (i obrnuto).
Često po inerciji u glagolima od 3 m\=ore naglašavamo 2. m\=oru, te dobijemo \textit{nakadaka} tamo gdje ga ne bismo trebali imati.
Teško je unaprijed znati koji glagoli su koji; stoga je jedino rješenje tog problema naučiti njihov naglasak napamet.
Evo i primjera često korištenih \textit{heiban} glagola.

\begin{itemize}
	\begin{multicols}{3}
		\item \pitch{あ/0,け/1,る/1}(開ける\footnote{također i 空ける, 明ける})
		\item \pitch{あ/0,げ/1,る/1}(上げる)
		\item \pitch{あ/0,が/1,る/1}(上がる)
		\item \pitch{の/0,ぼ/1,る/1}(昇る\footnote{također i 上る, 登る})
		\item \pitch{い/0,れ/1,る/1}(入れる)
		\item \pitch{は/0,じ/1,め/1,る/1}(始める\footnote{također i 初める})
		\item \pitch{は/0,じ/1,ま/1,る/1}(始まる)
		\item \pitch{と/0,め/1,る/1}(止める\footnote{također i 停める})
		\item \pitch{と/0,ま/1,る/1}(止まる)
		\item \pitch{と/0,ま/1,る/1}(泊まる)
		\item \pitch{や/0,め/1,る/1}(辞める)
		\item \pitch{い/0,く/1}(行く)
		\item \pitch{き/0,く/1}(聞く\footnote{također i 聴く})
		\item \pitch{き/0,こ/1,え/1,る/1}(聞こえる)
		\item \pitch{き/1,る/0}(切る\textsuperscript{$\star$})
		\item \pitch{お/0,し/1,え/1,る/1}(教える)
		\item \pitch{あ/0,ら/1,う/1}(洗う)
		\item \pitch{か/0,ん/1,じ/1,る/1}(感じる)
		\item \pitch{あ/0,そ/1,ぶ/1}(遊ぶ)
		\item \pitch{つ/0,づ/1,く/1}(続く)
		\item \pitch{つ/0,づ/1,け/1,る/1}(続ける)
		\item \pitch{か/0,わ/1,る/1}(変わる)
		\item \pitch{か/0,わ/1,る/1}(代わる)
		\item \pitch{か/0,わ/1,る/1}(替わる)
		\item \pitch{か/0,わ/1,る/1}(換わる)
		\item \pitch{か/0,え/1,る/1}(変える\textsuperscript{$\star$})
		\item \pitch{は/0,く/1}(履く\textsuperscript{$\star$})
		\item \pitch{あ/0,び/1,る/1}(浴びる)
		\item \pitch{わ/0,ら/1,う/1}(笑う)
		\item \pitch{な/0,く/1}(泣く)
		\item \pitch{な/0,く/1}(鳴く)
		\item \pitch{か/0,り/1,る/1}(借りる)
		\item \pitch{き/0,え/1,る/1}(消える)
		\item \pitch{く/0,ら/1,べ/1,る/1}(比べる)
		\item \pitch{す/0,わ/1,る/1}(座る)
		\item \pitch{き/0,め/1,る/1}(決める)
		\item \pitch{き/0,ま/1,る/1}(決まる)
		\item \pitch{の/0,せ/1,る/1}(乗せる)
		\item \pitch{の/0,る/1}(乗る)
		\item \pitch{お/0,わ/1,る/1}(終わる)
		\item \pitch{ま/0,げ/1,る/1}(曲げる)
		\item \pitch{ま/0,が/1,る/1}(曲がる)
		\item \pitch{わ/0,た/1,す/1}(渡す)
		\item \pitch{わ/0,た/1,る/1}(渡る)
		\item \pitch{ま/0,わ/1,す/1}(回す)
		\item \pitch{ま/0,わ/1,る/1}(回る\footnote{također i 周る})
		\item \pitch{わ/0,す/1,れ/1,る/1}(忘れる)
		\item \pitch{し/0,ぬ/1}(死ぬ)
		\item \pitch{く/0,れ/1,る/1}(呉れる)
		\item \pitch{も/0,ら/1,う/1}(貰う)
		\item \pitch{す/0,る/1}
		\item \pitch{や/0,る/1}
		\item \pitch{し/0,ま/1,う/1}
	\end{multicols}
\end{itemize}

U fusnotama su za neke glagole navedeni alternativni načini pisanja, kod kojih je razlika u značenju uglavnom suptilna i očituje se samo u pisanom jeziku.
Nešto kao regionalne varijante ovih novih pokemona, oni s dugim vratovima.
Svi oni imaju isti tip naglaska.
Poneki glagoli su homonimi, npr. 鳴く i 泣く, koji imaju isti izgovor (uključujući naglasak), ali različito značenje.
Međutim, neki od glagola (označeni zvjezdicom\textsuperscript{$\star$}) imaju homonimne parove koji se samo igrom slučaja sastoje od istih m\=ora, a naglasak je drugačiji, kao i njihovo značenje.
Dolje su navedeni i njihovi pripadni parovi:

\begin{itemize}
	\item \pitch{は/0,く/1}(履く) vs. \pitch{は/1,く/0}(吐く、掃く)
	\item \pitch{き/0,る/1}(着る) vs. \pitch{き/1,る/0}(切る)
	\item \pitch{か/0,え/1,る/1}(変える) vs.  \pitch{か/1,え/0,る/0}(帰る)
\end{itemize}

Ono što se još može primijetiti je da prijelazno-neprijelazni parovi imaju isti tip naglaska, npr. 続ける --- 続く、決める --- 決まる、回る --- 回す su svi \textit{heiban}.
Ovo vrijedi i za glagole drugih naglasnih tipova.
Iznimka su parovi kod kojih je jedan glagol od 2 m\=ore (取る --- 取れる、折る --- 折れる、立つ --- 立てる, itd.), te često korišteni glagoli 入る --- 入れる.

\subsubsection*{\textit{Atamadaka} glagoli}
\textit{Atamadaka} su relativno laki za zapamtiti, pošto ih nema toliko puno kao i drugih tipova.
Dodatno, ti glagoli su najčešće od 2 m\=ore, rjeđe od 3 ili više, i njihov naglasak sjeda na mjesto gdje bi se donekle i očekivalo.
Evo nekih često korištenih primjera:

\begin{itemize}
	\begin{multicols}{3}
		\item \pitch{の/1,む/0}(飲む)
		\item \pitch{み/1,る/0}(見る\footnote{također i 診る、観る、視る})
		\item \pitch{あ/1,る/0}(ある)
		\item \pitch{よ/1,む/0}(読む)
		\item \pitch{で/1,る/0}(出る)
		\item \pitch{く/1,る/0}(来る)
		\item \pitch{く/1,る/0}(繰る)
		\item \pitch{あ/1,う/0}(会う)
		\item \pitch{あ/1,う/0}(合う)
		\item \pitch{か/1,く/0}(書く)
		\item \pitch{か/1,え/0,る/0}(帰る)
		\item \pitch{か/1,え/0,す/0}(返す)
		\item \pitch{と/1,お/0,る/0}(通る)
		\item \pitch{と/1,お/0,す/0}(通す)
		\item \pitch{は/1,い/0,る/0}(入る)
	\end{multicols}
\end{itemize}

\subsubsection*{\textit{Nakadaka} glagoli}
Kako je već spomenuto, teško je na temelju intuicije pogoditi je li neki glagol \textit{heiban} ili \textit{nakadaka}.
Ne pomaže ni činjenica da su po učestalosti otprilike jednako česti kao i \textit{heiban} glagoli.
Iako zvuči prirodno ako izgovaramo sve glagole kao da su \textit{nakadaka}, u stvarnosti to samo neki jesu.
Primjeri \textit{nakadaka} glagola:

\begin{itemize}
	\begin{multicols}{3}
		\item \pitch{た/0,べ/1,る/0}(食べる)
		\item \pitch{わ/0,か/1,る/0}(分かる)
		\item \pitch{は/0,な/1,す/0}(話す)
		\item \pitch{し/0,ん/1,じ/1,る/0}(信じる)
		\item \pitch{は/0,し/1,る/0}(走る)
		\item \pitch{か/0,ん/1,が/1,え/0,る/0}(考える)
		\item \pitch{つ/0,く/1,る/0}(作る)
		\item \pitch{み/0,え/1,る/0}(見える)
		\item \pitch{み/0,せ/1,る/0}(見せる)
		\item \pitch{い/0,き/1,る/0}(生きる)
		\item \pitch{に/0,げ/1,る/0}(逃げる)
		\item \pitch{の/0,が/1,す/0}(逃す)
		\item \pitch{の/0,が/1,れ/1,る/0}(逃れる)
		\item \pitch{た/0,す/1,け/1,る/0}(助ける)
		\item \pitch{た/0,す/1,か/1,る/0}(助かる)
		\item \pitch{し/0,ら/1,べ/1,る/0}(調べる)
		\item \pitch{お/0,も/1,う/0}(思う)
		\item \pitch{あ/0,つ/1,め/1,る/0}(集める)
		\item \pitch{あ/0,つ/1,ま/1,る/0}(集まる)
		\item \pitch{う/0,ご/1,く/0}(動く)
		\item \pitch{う/0,ご/1,か/1,す/0}(動かす)
		\item \pitch{も/0,ど/1,る/0}(戻る)
		\item \pitch{も/0,ど/1,す/0}(戻す)
		\item \pitch{お/0,ぼ/1,え/1,る/0}(覚える)
		\item \pitch{さ/0,め/1,る/0}(覚める)
		\item \pitch{ひ/0,え/1,る/0}(冷える)
		\item \pitch{ひ/0,や/1,す/0}(冷やす)
		\item \pitch{お/0,よ/1,ぐ/0}(泳ぐ)
		\item \pitch{あ/0,た/1,た/1,め/1,る/0}(温める\footnote{također i 暖める})
		\item \pitch{あ/0,た/1,た/1,ま/1,る/0}(温まる\footnote{također i 暖める})
		\item \pitch{さ/0,け/1,ぶ/0}(叫ぶ)
		\item \pitch{よ/0,ろ/1,こ/1,ぶ/0}(喜ぶ)
		\item \pitch{え/0,ら/1,ぶ/0}(選ぶ)
		\item \pitch{こ/0,ま/1,る/0}(困る)
		\item \pitch{あ/0,ま/1,る/0}(余る)
		\item \pitch{わ/0,か/1,れ/1,る/0}(別れる)
		\item \pitch{お/0,れ/1,る/0}(折れる)
		\item \pitch{と/0,れ/1,る/0}(取れる)
		\item \pitch{な/0,お/1,る/0}(治る\footnote{također i 直る})
		\item \pitch{な/0,お/1,す/0}(治す\footnote{također i 直す})
		\item \pitch{か/0,け/1,る/0}(掛ける)
		\item \pitch{か/0,か/1,る/0}(掛かる)
		\item \pitch{た/0,お/1,す/0}(倒す)
		\item \pitch{た/0,お/1,れ/1,る/0}(倒れる)
		\item \pitch{ゆ/0,る/1,す/0}(許す)
		\item \pitch{こ/0,と/1,な/1,る/0}(異なる)
		\item \pitch{か/0,な/1,う/0}(叶う)
		\item \pitch{は/0,な/1,れ/1,る/0}(離れる)
		\item \pitch{は/0,な/1,す/0}(離す)
		\item \pitch{す/0,ぎ/1,る/0}(過ぎる)
		\item \pitch{ま/0,じ/1,る/0}(交じる)
		\item \pitch{ま/0,じ/1,る/0}(混じる)
		\item \pitch{ま/0,ぜ/1,る/0}(交ぜる)
		\item \pitch{ま/0,ぜ/1,る/0}(混ぜる)
		\item \pitch{で/0,き/1,る/0}
		\item \pitch{く/0,だ/1,さ/1,る/0}
		\item \pitch{な/0,さ/1,る/0}
	\end{multicols}
\end{itemize}

\subsection{Naglasak kod pristojnih oblika glagola}
Pristojni 〜ます oblici glagola istog su naglasnog tipa za sve glagole, što ih čini vrlo lakima za zapamtiti.
Pri spajanju s i-osnovom glagola, pomoćni glagol \pitch{ま/1,す/0} te njegovi izvedeni oblici \pitch{ま/1,し/0,た/0} i \pitch{ま/0,せ/1,ん/0} gaze naglasak prvog glagola.
Pri tvorbi negativnog prošlog vremena kopula でした čini naglasnu cjelinu s prethodnim glagolom i naglasak pritom ne mijenja mjesto.
Primjeri:
\begin{itemize}
	\item \textit{atamadaka} glagol \pitch{の/1,む/0}(飲む):
	\begin{itemize}
		\begin{multicols}{2}
			\item \pitch{の/0,み/1,ま/1,す/0}
			\item \pitch{の/0,み/1,ま/1,せ/1,ん/0}
			\item \pitch{の/0,み/1,ま/1,し/0,た/0}
			\item \pitch{の/0,み/1,ま/1,せ/1,ん/0,で/0,し/0,た/0}
		\end{multicols}
	\end{itemize}
	\item \textit{nakadaka} glagol \pitch{た/0,べ/1,る/0}(食べる):
	\begin{itemize}
		\begin{multicols}{2}
			\item \pitch{た/0,べ/1,ま/1,す/0}
			\item \pitch{た/0,べ/1,ま/1,せ/1,ん/0}
			\item \pitch{た/0,べ/1,ま/1,し/0,た/0}
			\item \pitch{た/0,べ/1,ま/1,せ/1,ん/0,で/0,し/0,た/0}
		\end{multicols}
	\end{itemize}
	\item \textit{heiban} glagol \pitch{あ/0,け/1,る/1}(開ける):
	\begin{itemize}
		\begin{multicols}{2}
			\item \pitch{あ/0,け/1,ま/1,す/0}
			\item \pitch{あ/0,け/1,ま/1,せ/1,ん/0}
			\item \pitch{あ/0,け/1,ま/1,し/0,た/0}
			\item \pitch{あ/0,け/1,ま/1,せ/1,ん/0,で/0,し/0,た/0}
		\end{multicols}
	\end{itemize}
\end{itemize}
\subsection*{Coming up next...}
\subsection{Naglasak kod negativnih oblika glagola}
\subsection{Naglasak kod ostalih oblika glagola}
