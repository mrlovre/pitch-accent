% !TeX root = document.tex
Svi negativni oblici u japanskom, uključujući oblike imenica, pridjeva i glagola, nezaobilazno se tvore pomoću idjeva \pitch{な/1,い/0}(無い), koji je istovremeno još jedan od \textit{atamadaka} idjeva i znači ``nepostojeći''.
Naglasci njegovih izvedenih oblika su također \textit{atamadaka}: \pitch{な/1,く/0}、\pitch{な/1,さ/0}、 \pitch{な/1,か/0,っ/0,た/0}、\pitch{な/1,く/0,て/0}、\pitch{な/1,け/0,れ/0,ば/0}.

\subsection{Negativni oblici imenica i nadjeva}
Kod negativnih oblika imenica i nadjeva, potrebna je još i kopula で + čestica は, ili njihov skraćeni oblik じ\m{ゃ}. One se stapaju s imenicom (odnosno nadjevom) u jednu naglasnu cjelinu prema pravilu \textit{gaženja}.
Pomoćni idjev ない čini zasebnu naglasnu cjelinu.
Primjeri (prekid u liniji označava granicu naglasne cjeline, ali ne upućuje na pauzu između riječi):
\begin{itemize}
	\item \textit{heiban} imenica, \pitch{が/0,く/1,せ/1,い/1}(学生):
	\begin{itemize}
		\item \pitch{が/0,く/1,せ/1,い/1}+\pitch{じ\m{ゃ}/0}+\pitch{な/1,い/0}=\pitch{が/0,く/1,せ/1,い/1,じ\m{ゃ}/1}\!\pitch{な/1,い/0}(学生じ\m{ゃ}ない)
		\item \pitch{が/0,く/1,せ/1,い/1}+\pitch{で/1,は/0}+\pitch{な/1,い/0}=\pitch{が/0,く/1,せ/1,い/1,で/1,は/0}\!\pitch{な/1,い/0}(学生ではない)
		\item \pitch{が/0,く/1,せ/1,い/1}+\pitch{じ\m{ゃ}/0}+\pitch{な/1,か/0,っ/0,た/0}=\pitch{が/0,く/1,せ/1,い/1,じ\m{ゃ}/1}\!\pitch{な/1,か/0,っ/0,た/0}\\(学生じ\m{ゃ}なかった)
		\item \pitch{が/0,く/1,せ/1,い/1}+\pitch{で/1,は/0}+\pitch{な/1,か/0,っ/0,た/0,/0}=\pitch{が/0,く/1,せ/1,い/1,で/1,は/0}\!\pitch{な/1,か/0,っ/0,た/0}\\(学生ではなかった)
	\end{itemize}
	\item \textit{nakadaka} imenica, \pitch{せ/0,ん/1,せ/1,い/0}(先生):
	\begin{itemize}
		\item \pitch{せ/0,ん/1,せ/1,い/0}+\pitch{じ\m{ゃ}/0}+\pitch{な/1,い/0}=\pitch{せ/0,ん/1,せ/1,い/0,じ\m{ゃ}/0}\!\pitch{な/1,い/0}(先生じゃない)
		\item \pitch{せ/0,ん/1,せ/1,い/0}+\pitch{で/1,は/0}+\pitch{な/1,い/0}=\pitch{せ/0,ん/1,せ/1,い/0,で/0,は/0}\!\pitch{な/1,い/0}(先生ではない)
		\item \pitch{せ/0,ん/1,せ/1,い/0}+\pitch{じ\m{ゃ}/0}+\pitch{な/1,か/0,っ/0,た/0}=\pitch{せ/0,ん/1,せ/1,い/0,じ\m{ゃ}/0}\!\pitch{な/1,か/0,っ/0,た/0}\\(先生じ\m{ゃ}なかった)
		\item \pitch{せ/0,ん/1,せ/1,い/0}+\pitch{で/1,は/0}+\pitch{な/1,か/0,っ/0,た/0}=\pitch{せ/0,ん/1,せ/1,い/0,で/0,は/0}\!\pitch{な/1,か/0,っ/0,た/0}\\(先生ではなかった)
	\end{itemize}
	\item \textit{heiban} nadjev, \pitch{か/0,ん/1,た/1,ん/1}(簡単):
	\begin{itemize}
		\item \pitch{か/0,ん/1,た/1,ん/1}+\pitch{じ\m{ゃ}/0}+\pitch{な/1,い/0}=\pitch{か/0,ん/1,た/1,ん/1,じ\m{ゃ}/1}\!\pitch{な/1,い/0}(簡単じ\m{ゃ}ない)
		\item \pitch{か/0,ん/1,た/1,ん/1}+\pitch{で/1,は/0}+\pitch{な/1,い/0}=\pitch{か/0,ん/1,た/1,ん/1,で/1,は/0}\!\pitch{な/1,い/0}(簡単ではない)
		\item \pitch{か/0,ん/1,た/1,ん/1}+\pitch{じ\m{ゃ}/0}+\pitch{な/1,か/0,っ/0,た/0}=\pitch{か/0,ん/1,た/1,ん/1,じ\m{ゃ}/1}\!\pitch{な/1,か/0,っ/0,た/0}\\(簡単じ\m{ゃ}なかった)
		\item \pitch{か/0,ん/1,た/1,ん/1}+\pitch{で/1,は/0}+\pitch{な/1,か/0,っ/0,た/0,/0}=\pitch{か/0,ん/1,た/1,ん/1,で/1,は/0}\!\pitch{な/1,か/0,っ/0,た/0}\\(簡単ではなかった)
	\end{itemize}
	\item \textit{atamadaka} nadjev, \pitch{げ/1,ん/0,き/0}(元気):
	\begin{itemize}
		\item \pitch{げ/1,ん/0,き/0}+\pitch{じ\m{ゃ}/0}+\pitch{な/1,い/0}=\pitch{げ/1,ん/0,き/0,じ\m{ゃ}/0}\!\pitch{な/1,い/0}(元気じ\m{ゃ}ない)
		\item \pitch{げ/1,ん/0,き/0}+\pitch{で/1,は/0}+\pitch{な/1,い/0}=\pitch{げ/1,ん/0,き/0,で/0,は/0}\!\pitch{な/1,い/0}(元気ではない)
		\item \pitch{げ/1,ん/0,き/0}+\pitch{じ\m{ゃ}/0}+\pitch{な/1,い/0}=\pitch{げ/1,ん/0,き/0,じ\m{ゃ}/0}\!\pitch{な/1,か/0,っ/0,た/0}\\(元気じ\m{ゃ}なかった)
		\item \pitch{げ/1,ん/0,き/0}+\pitch{で/1,は/0}+\pitch{な/1,か/0,っ/0,た/0}=\pitch{げ/1,ん/0,き/0,で/0,は/0}\!\pitch{な/1,か/0,っ/0,た/0}\\(元気ではなかった)
	\end{itemize}
\end{itemize}

Pristojne verzije negativnih oblika imenica i nadjeva dobiju se korištenjem ます oblika glagola ある, odnosno njegove negacije, umjesto pomoćnog idjeva ない.
Njihovi naglasni oblici su sljedeći\footnote{više o naglasku kod glagolskih oblika u odjeljku \ref{sec:glagoli}}:
\begin{itemize}
	\begin{multicols}{2}
		\item \pitch{あ/0,り/1,ま/1,せ/1,ん/0}
		\item \pitch{あ/0,り/1,ま/1,せ/1,ん/0,で/0,し/0,た/0}
	\end{multicols}
\end{itemize}
Pri spajanju s imenicama i nadjevima oni čine zasebne naglasne cjeline.
Primjeri:
\begin{itemize}
	\item \pitch{せ/0,ん/1,せ/1,い/0}+\pitch{じ\m{ゃ}/0}+\pitch{あ/0,り/1,ま/1,せ/1,ん/0}=\pitch{せ/0,ん/1,せ/1,い/0,じ\m{ゃ}/0}\!\pitch{あ/0,り/1,ま/1,せ/1,ん/0}\\(先生じゃありません)
	\item \pitch{せ/0,ん/1,せ/1,い/0}+\pitch{で/1,は/0}+\pitch{あ/0,り/1,ま/1,せ/1,ん/0}=\pitch{せ/0,ん/1,せ/1,い/0,で/0,は/0}\!\pitch{あ/0,り/1,ま/1,せ/1,ん/0}\\(先生ではありません)
	\item \pitch{せ/0,ん/1,せ/1,い/0}+\pitch{じ\m{ゃ}/0}+\pitch{あ/0,り/1,ま/1,せ/1,ん/0,で/0,し/0,た/0}=\pitch{せ/0,ん/1,せ/1,い/0,じ\m{ゃ}/0}\!\pitch{あ/0,り/1,ま/1,せ/1,ん/0,で/0,し/0,た/0}\\(先生じ\m{ゃ}ありませんでした)
	\item \pitch{せ/0,ん/1,せ/1,い/0}+\pitch{で/1,は/0}+\pitch{あ/0,り/1,ま/1,せ/1,ん/0,で/0,し/0,た/0}=\pitch{せ/0,ん/1,せ/1,い/0,で/0,は/0}\!\pitch{あ/0,り/1,ま/1,せ/1,ん/0,で/0,し/0,た/0}\\(先生ではありませんでした)
\end{itemize}

\subsection{Negativni oblici idjeva}
Kod negativnih oblika idjeva, pomoćni idjev ない dodaje se na 〜く oblik, i ovaj puta se stapa sa naglasnom cjelinom osnovne riječi, pri čemu vrijedi pravilo \textit{gaženja}.
Primjeri:
\begin{itemize}
	\item \textit{heiban} idjev, \pitch{あ/0,か/1,い/1}(赤い):
	\begin{itemize}
		\item \pitch{あ/0,か/1,く/1}+\pitch{な/1,い/0}=\pitch{あ/0,か/1,く/1,な/1,い/0}(赤くない)
		\item \pitch{あ/0,か/1,く/1}+\pitch{な/1,か/0,っ/0,た/0}=\pitch{あ/0,か/1,く/1,な/1,か/0,っ/0,た/0}(赤くなかった)
	\end{itemize}
	\item \textit{nakadaka} idjev (od 3 m\=ore), \pitch{あ/0,つ/1,い/0}(暑い):
	\begin{itemize}
		\item \pitch{あ/1,つ/0,く/0}+\pitch{な/1,い/0}=\pitch{あ/1,つ/0,く/0,な/0,い/0}(暑くない)
		\item \pitch{あ/1,つ/0,く/0}+\pitch{な/1,か/0,っ/0,た/0}=\pitch{あ/1,つ/0,く/0,な/0,か/0,っ/0,た/0}(暑くなかった)
	\end{itemize}
	\item \textit{nakadaka} idjev (do 4 ili više m\=ora), \pitch{お/0,も/1,し/1,ろ/1,い/0}(面白い):
	\begin{itemize}
		\item \pitch{お/0,も/1,し/1,ろ/1,く/0}+\pitch{な/1,い/0}=\pitch{お/0,も/1,し/1,ろ/1,く/0,な/0,い/0}(面白くない)
		\item \pitch{お/0,も/1,し/1,ろ/1,く/0}+\pitch{な/1,か/0,っ/0,た/0}=\pitch{お/0,も/1,し/1,ろ/1,く/0,な/0,か/0,っ/0,た/0}(面白くなかった)
	\end{itemize}
\end{itemize}

Što se tiče pristojnih verzija negativnih oblika idjeva, uvijek treba dodati samo pristojnu verziju kopule です na kraju.
Ona se spaja u jednu naglasnu cjelinu s negativnim oblikom i zadržava naglasak na mjestu na kojem je bio, npr.:
\begin{itemize}
	\item \pitch{あ/1,つ/0,く/0,な/0,い/0,で/0,す/0}(暑くないです)
	\item \pitch{あ/1,つ/0,く/0,な/0,か/0,っ/0,た/0,で/0,す/0}(暑くなかったです)
\end{itemize}
