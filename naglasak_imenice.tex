% !TeX root = document.tex
Imenice mogu imati bilo koji od 4 naglaska, kao što je pokazano na nizu primjera u prethodnom odjeljku.
Specijalni slučaj čine dugačke imenice koje su nastale srastanjem dviju ili više drugih imenica.
Njihov naglasni tip je gotovo uvijek \textit{nakadaka}, i to takav da je naglasak na prvoj mori druge (ili posljednje) riječi u nizu.
Pritom nije bitno kakav naglasak imaju individualne riječi.

Primjeri:
\begin{itemize}
	\item \pitch{や/0,ま/1,が/0}(山が)+\pitch{の/0,ぼ/1,り/1,が/1}(登りが)=\pitch{や/0,ま/1,の/1,ぼ/0,り/0}(山登り)
	\item \pitch{は/0,く/1,じ/1,が/1}(白人が)+\pitch{だ/0,ん/1,せ/1,い/1,が/1}(男性が)=%
	\begin{minipage}{3cm}\pitch{は/0,く/1,じ/1,ん/1,だ/0,ん/0,せ/0,い/0}\\(白人男性)\end{minipage}
\end{itemize}

Međutim, ima i riječi koje ne slijede gore navedeno pravilo o naglasku na početku druge riječi.
Riječi za nacionalnost sa sufiksom 人 uglavnom imaju naglasak jednu m\=oru prije じ iz 人, osim u slučaju kada ta m\=ora ne može nositi naglasak (日本, スペイン).
Riječi za jezik sa sufiksom 語 su \textit{heiban} bez obzira na naglasak prve riječi.
U tablici \ref{tab:drzava} navedeni su neki primjeri država, nacionalnosti i jezika.

\begin{table}[htbp]
	\renewcommand{\arraystretch}{2}
	\centering
	\caption[]{Primjeri država, jezika i nacionalnosti}
	\label{tab:drzava}
	\begin{tabular}{llll}
		\toprule
		& \multicolumn{1}{c}{国} & \multicolumn{1}{c}{人} & \multicolumn{1}{c}{語} \\
		\midrule
		日本 & \pitch{に/0,ほ/1,ん/0} & \pitch{に/0,ほ/1,ん/1,じ/1,ん/0} & \pitch{に/0,ほ/1,ん/1,ご/1} \\
		中国 & \pitch{ち\m{ゅ}/1,う/0,ご/0,く/0} & \pitch{ち\m{ゅ}/0,う/1,ご/1,く/1,じ/0,ん/0} & \pitch{ち\m{ゅ}/0,う/1,ご/1,く/1,ご/1} \\
		韓国 & \pitch{か/1,ん/0,こ/0,く/0} & \pitch{か/0,ん/1,こ/1,く/1,じ/0,ん/0} & \pitch{か/0,ん/1,こ/1,く/1,ご/1} \\
		ロシア & \pitch{ろ/1,し/0,あ/0} & \pitch{ろ/0,し/1,あ/1,じ/0,ん/0} & \pitch{ろ/0,し/1,あ/1,ご/1} \\
		ドイツ & \pitch{ど/1,い/0,つ/0} & \pitch{ど/0,い/1,つ/1,じ/0,ん/0} & \pitch{ど/0,い/1,つ/1,ご/1} \\
		スペイン & \pitch{す/0,ぺ/1,い/0,ん/0} & \pitch{す/0,ぺ/1,い/1,ん/0,じ/0,ん/0} & \pitch{す/0,ぺ/1,い/1,ん/1,ご/1} \\
		フランス & \pitch{ふ/0,ら/1,ん/1,す/1} & \pitch{ふ/0,ら/1,ん/1,す/1,じ/0,ん/0} & \pitch{ふ/0,ら/1,ん/1,す/1,ご/1} \\
		ポーランド & \pitch{ぽ/1,ー/0,ら/0,ん/0,ど/0} & \pitch{ぽ/0,ー/1,ら/1,ん/1,ど/1,じ/0,ん/0} & \pitch{ぽ/0,ー/1,ら/1,ん/1,ど/1,ご/1} \\
		クロアチア & \pitch{く/0,ろ/1,あ/1,ち/1,あ/1} & \pitch{く/0,ろ/1,あ/1,ち/1,あ/1,じ/0,ん/0} & \pitch{く/0,ろ/1,あ/1,ち/1,あ/1,ご/1} \\
		セルビア & \pitch{せ/0,る/1,び/1,あ/1} & \pitch{せ/0,る/1,び/1,あ/1,じ/0,ん/0} & \pitch{せ/0,る/1,び/1,あ/1,ご/1} \\
		ボスニア & \pitch{ぼ/0,す/1,に/1,あ/1} & \pitch{ぼ/0,す/1,に/1,あ/1,じ/0,ん/0} & \pitch{ぼ/0,す/1,に/1,あ/1,ご/1} \\
		アラビア & \pitch{あ/0,ら/1,び/1,あ/1} & \pitch{あ/0,ら/1,び/1,あ/1,じ/0,ん/0} & \pitch{あ/0,ら/1,び/1,あ/1,ご/1} \\
		\makecell[tl]{ナメック星\\(Namek)} & \pitch{な/0,め/1,っ/1,く/1,せ/1,い/1} & \pitch{な/0,め/1,っ/1,く/1,せ/1,い/0,じ/0,ん/0} & \pitch{な/0,め/1,く/1,ご/1} \\
		\makecell[tl]{クロノシュ\\(Q'onoS)} & \pitch{く/0,ろ/1,の/1,し\m{ゅ}/1} & \pitch{く/0,り/1,ん/1,ご/1,ん/0,じ/0,ん/0} & \pitch{く/0,り/1,ん/1,ご/1,ん/1,ご/1} \\
		\bottomrule
	\end{tabular}
	\renewcommand{\arraystretch}{1.25}
\end{table}


\subsection{Naglasak u čestim kombinacijama imenica s drugim riječima}
Dosad razmatrani primjeri uključivali su po jednu jednostavnu ili složenu imenicu koja sama za sebe čini naglasnu cjelinu.
No, imenice se spajaju i s kopulama i različitim sufiksima, čineći opet jednu naglasnu cjelinu koja ima promijenjen naglasak u odnosu na individualne riječi od kojih je nastala.
Nasreću, ove promjene se doagađaju po predvidivim, relativno jednostavnim pravilima.

\subsubsection*{Spajanje imenica s kopulama だ i です}
Spajanjem imenice s kopulom だ nastaje naglasna cjelina čiji naglasak ovisi o naglasku imenice, na sljedeći način:
\begin{enumerate}
	\item Ako imenica nema naglasak (\textit{heiban}), tada naglasna cjelina opet nema naglasak
	\begin{itemize}
		\item \pitch{が/0,く/1,せ/1,い/1}+\pitch{だ/0}=\pitch{が/0,く/1,せ/1,い/1,だ/1}(学生)
	\end{itemize}
	\item Ako imenica ima naglasak (ostali tipovi), tada naglasna cjelina ima naglasak na istom mjestu na kojem je bio u imenici
	\begin{itemize}
		\item \textit{atamadaka} --- \pitch{じ/1,し\m{ょ}/0}+\pitch{だ/0}=\pitch{じ/1,し\m{ょ}/0,だ/0}(辞書)
		\item \textit{nakadaka} --- \pitch{せ/1,ん/1,せ/1,い/0}+\pitch{だ/0}=\pitch{せ/0,ん/1,せ/1,い/0,だ/0}(先生)
		\item \textit{odaka} --- \pitch{お/0,と/1,こ/1}+\pitch{だ/0}=\pitch{お/0,と/1,こ/1,だ/0}(男)
	\end{itemize}
\end{enumerate}

S druge strane, spajanje imenice s kopulom です uvodi male promjene u naglasak:
\begin{enumerate}
	\item Ako imenica nema naglasak (\textit{heiban}), tada naglasna cjelina ima naglasak na mori で u です:
	\begin{itemize}
		\item \pitch{が/0,く/1,せ/1,い/1}+\pitch{で/1,す/0}=\pitch{が/0,く/1,せ/1,い/1,で/1,す/0}(学生)
	\end{itemize}
	\item Ako imenica ima naglasak (ostali tipovi), tada naglasna cjelina ima naglasak na istom mjestu na kojem je bio u imenici
	\begin{itemize}
		\item \textit{atamadaka} --- \pitch{じ/1,し\m{ょ}/0}+\pitch{で/1,す/0}=\pitch{じ/1,し\m{ょ}/0,で/0,す/0}(辞書)
		\item \textit{nakadaka} --- \pitch{せ/1,ん/1,せ/1,い/0}+\pitch{で/1,す/0}=\pitch{せ/0,ん/1,せ/1,い/0,で/0,す/0}(先生)
		\item \textit{odaka} --- \pitch{お/0,と/1,こ/1}+\pitch{で/1,す/0}=\pitch{お/0,と/1,こ/1,で/0,す/0}(男)
	\end{itemize}
\end{enumerate}

Iako se kopule \pitch{だ/0} i \pitch{で/1,す/0} ne koriste kao samostalne riječi, u svrhu njihovog spajanja na imenice možemo si pomoći ako ih promatramo kao \textit{heiban}, odnosno \textit{atamadaka}.
Tada se gornje promjene mogu opisati samo jednim pravilom (koje vrijedi i šire od spajanja imenica s kopulom): \textit{atamadaka}, \textit{nakadaka} i \textit{odaka} \textit{gaze}\footnote{engl. \textit{override}, termin posuđen iz koncepta objektno orijentiranog programiranja koji u današnje doba, kada jezici kao što je Python dobivaju na popularnosti i rugaju se sa smislom života, doživljava sve veću propast} \textit{heiban}.
Možemo to promatrati i na još jedan način: \textit{heiban} je u stvari odsutnost naglaska, a budući da naglasna cjelina može imati najviše jedan naglasak, ostaje onaj naglasak koji je došao prvi po redu.
{\color{white} {Treći način (hard-core): skup naglasaka, asocijativna operacija 
$f(\text{平板}, b) = b,\ f\left(a, \color{black}\cdot\color{white}\right) = a$, i neutralni element 平板 čine monoid.}}

\subsubsection*{Spajanje imenica s česticama}
Često korištene čestice od jedne m\=ore, \pitch{は/0}、\pitch{も/0}、\pitch{が/0}、\pitch{を/0}、\pitch{に/0}、\pitch{で/0}、\pitch{と/0} nikad se ne koriste samostalno, ali opet, u okviru njihovog povezivanja na imenice, možemo smatrati da su \textit{heiban}.
U tom slučaju naglasak će ostati na istom mjestu nakon što se na imenicu nalijepi neka čestica.

Zapravo, ranije je razlika između \textit{heiban} i \textit{odaka} naglasnih tipova demonstrirana dodavanjem ovih čestica na kraj riječi upravo zahvaljujući ovom svojstvu, a to je da one ne mijenjaju mjesto naglaska unutar naglasne cijeline.

Na isti način, složene čestice \pitch{で/1,は/0}、\pitch{で/1,も/0}、\pitch{に/1,も/0}、\pitch{と/1,も/0}、\pitch{か/1,が/0}、\pitch{か/1,を/0}、\pitch{か/1,も/0} možemo promatrati kao da su \textit{odaka}.
Pravila o \textit{gaženju} naglasaka ostaju ista.

\subsubsection*{Spajanje imenica s ostalim čestim nastavcima}
Svršeni oblici kopule \pitch{だ/1,っ/0,た/0} i \pitch{で/1,し/0,た/0}, volicionalni oblici \pitch{だ/0,ろ/1,う/0} i \pitch{で/0,し\m{ょ}/1,う/0}, te hipotetski oblik \pitch{な/1,ら/0,ば/0} mogu se uzeti kao da imaju naglasak kako im je naznačeno.
Pri spajanju s \textit{heiban} riječima cjelina preuzima naglasak kopule, a pri spajanju s ostalim tipovima naglasaka naglasak ostaje na riječi s kojom se spaja.
Primjeri povezivanja sa だろう i でし\!ょう (na isti način se povezuju i だった、でした、ならば):
\begin{itemize}
	\item \textit{heiban} --- \pitch{が/0,く/1,せ/1,い/1,だ/1,ろ/1,う/0}、\pitch{が/0,く/1,せ/1,い/1,で/1,し\m{ょ}/1,う/0}(学生)
	\item \textit{atamadaka} --- \pitch{じ/1,し\m{ょ}/0,だ/0,ろ/0,う/0}、\pitch{じ/1,し\m{ょ}/0,で/0,し\m{ょ}/0,う/0}(辞書)
	\item \textit{nakadaka} --- \pitch{せ/0,ん/1,せ/1,い/0,だ/0,ろ/0,う/0}、\pitch{せ/0,ん/1,せ/1,い/0,で/0,し\m{ょ}/0,う/0}(先生)
	\item \textit{odaka} --- \pitch{お/0,と/1,こ/1,だ/0,ろ/0,う/0}、\pitch{お/0,と/1,こ/1,で/0,し\m{ょ}/0,う/0}(男)
\end{itemize}

Također, često se koriste i \pitch{み/1,た/0,い/0} te \pitch{ら/0,し/1,い/0} kada znači da se radi o govornikovom zaključku temeljenom na nečemu što je čuo, vidio, ili pročitao.
Vrijede ista pravila spajanja kao i za dosad navedeno:
\begin{itemize}
	\item \textit{heiban} --- \pitch{が/0,く/1,せ/1,い/1,み/1,た/0,い/0}、\pitch{が/0,く/1,せ/1,い/1,ら/1,し/1,い/0}(学生)
	\item \textit{atamadaka} --- \pitch{じ/1,し\m{ょ}/0,み/0,た/0,い/0}、\pitch{じ/1,し\m{ょ}/0,ら/0,し/0,い/0}(辞書)
	\item \textit{nakadaka} --- \pitch{せ/0,ん/1,せ/1,い/0,み/0,た/0,い/0}、\pitch{せ/0,ん/1,せ/1,い/0,ら/0,し/0,い/0}(先生)
	\item \textit{odaka} --- \pitch{お/0,と/1,こ/1,み/0,た/0,い/0}、\pitch{お/0,と/1,こ/1,ら/0,し/0,い/0}(男)
\end{itemize}

Jedna često korištena iznimka je sufiks \pitch{ら/0,し/1,い/0} kada znači reprezentativnu karakteristiku nečeg\footnote{
	razlikujemo ga od prvog らしい po tome što prvi završava rečenicu (moguće i opisnu rečenicu) i mijenja njezin modalitet, a drugi je sufiks koji se dodaje na imenicu i ponaša se kao idjev pa može biti ispred imenice, ali prvi se isto ponaša kao idjev, znači ne razlikujemo ih}.
Razlikovati ih možemo jedino po smislu.
Pravilo da naglasna cjelina ima najviše jedan naglasak i dalje vrijedi, međutim ovaj sufiks uvijek gazi naglasak prve riječi:
\begin{itemize}
	\item \textit{heiban} --- \pitch{が/0,く/1,せ/1,い/1,ら/1,し/1,い/0}(学生)
	\item \textit{atamadaka} --- \pitch{じ/0,し\m{ょ}/1,ら/1,し/1,い/0}(辞書)
	\item \textit{nakadaka} --- \pitch{せ/0,ん/1,せ/1,い/1,ら/1,し/1,い/0}(先生)
	\item \textit{odaka} --- \pitch{お/0,と/1,こ/1,ら/1,し/1,い/0}(男)
\end{itemize}
Primijetimo da, u slučaju spajanja sa \textit{heiban} riječima, između prvog i drugog oblika nema razlike u naglasku.

Na isti način se ponašaju i nastavci \pitch{っ/0,ぽ/1,い/0}, \textit{nakadaka}, i \pitch{ふ/0,う/1}(風), \textit{heiban}, tj. njihovi naglasci uvijek gaze naglasak prve riječi.
