% !TeX root = document.tex
Pridjevi se u japanskom jeziku dijele u dvije vrste: neprave, imenske, na-pridjeve ili skraćeno --- nadjeve, te prave, i-pridjeve ili skraćeno --- idjeve.
Uloga obiju vrsta je opisivanje imenice, ali ove vrste se gramatički vrlo različito ponašaju.
Stoga će i njihovi naglasci biti različiti.

\subsection{Naglasak kod osnovnih oblika nadjeva}
Nadjevi su u osnovi podvrsta imenica s privilegijom da se mogu koristiti za opisivanje drugih imenica.
To znači da sve što je do sada vrijedilo za imenice vrijedi i za nadjeve.
Dodatno, nadjevi se spajaju sa opisnim oblikom kopule \pitch{な/0} na jednak način kao i imenice sa završnim oblik kopule \pitch{だ/0}, zajedno tvoreći naglasnu cjelinu.
Imenica koju nadjev opisuje pripada zasebnoj naglasnoj cjelini.
Iznimno, ako je naglasna cjelina pridjeva nenaglašena (\textit{heiban}), tada se može združiti s imenicom u novu naglasnu cjelinu koja ima naglasak na istom mjestu kao i ta imenica.
Primjeri (prekid u liniji označava granicu naglasne cjeline, ali ne upućuje na pauzu između riječi):
\begin{itemize}
	\item \pitch{す/0,て/1,き/1}+\pitch{な/1}+\pitch{せ/0,ん/1,せ/1,い/0}%
	\begin{minipage}[t]{8cm}
		=\pitch{す/0,て/1,き/1,な/1}\!\pitch{せ/0,ん/1,せ/1,い/0}(素敵な先生)\\
		\hspace*{0em}(\pitch{す/0,て/1,き/1,な/1,せ/1,ん/1,せ/1,い/0})
	\end{minipage}
	\item \pitch{げ/1,ん/0,き/0}+\pitch{な/0}+\pitch{が/0,く/1,せ/1,い/1}=\pitch{げ/1,ん/0,き/0,な/0}\!\pitch{が/0,く/1,せ/1,い/1}(元気な学生)
	\item \pitch{じ/0,ゆ/1,う/0}+\pitch{な/0}+\pitch{お/0,と/1,こ/1}=\pitch{じ/0,ゆ/1,う/0,な/0}\!\pitch{お/0,と/1,こ/1}(自由な男)
	\item \pitch{す/0,き/1,な/0}+\pitch{な/0}+\pitch{じ/1,し\m{ょ}/0}=\pitch{す/0,き/1,な/0}\!\pitch{じ/1,し\m{ょ}/0}(好きな辞書)
\end{itemize}

Tzv. prošli oblik nadjeva dobiva se spajanjem sa \pitch{だ/1,っ/0,た/0}.
Pritom nastala naglasna cjelina nikad nije \textit{heiban}, jer je だった naglašena.

Pravilo o riječima od 4 m\=ore koje se pišu s 2 kanđija vrijedi i za nadjeve.
Neke češće iznimke koje bi vrijedilo zapamtiti su:
\begin{itemize}
	\begin{multicols}{2}
		\item \pitch{て/1,い/0,ね/0,い/0,な/0}(丁寧な)
		\item \pitch{ね/1,っ/0,し/0,ん/0,な/0}(熱心な)
		\item \pitch{じゅ/0,う/1,ぶ/1,ん/0,な/0}(十分な)
		\item \pitch{し/1,ん/0,せ/0,つ/0,な/0,/0}(親切な)
	\end{multicols}
\end{itemize}

Evo i primjera učestalih nadjeva različitih naglasnih tipova:
\begin{enumerate}
	\item \textit{heiban}
	\begin{itemize}
		\begin{multicols}{3}
			\item \pitch{ひ/0,ま/1,な/1}(暇な)
			\item \pitch{む/0,だ/1,な/1}(無駄な)
			\item \pitch{す/0,て/1,き/1,な/1}(素敵な)
			\item \pitch{き/0,ら/1,い/1,な/1}(嫌いな)
			\item \pitch{じ\m{ょ}/0,う/1,ぶ/1,な/1}(丈夫な)
		\end{multicols}
	\end{itemize}
	\item \textit{atamadaka}
	\begin{itemize}
		\begin{multicols}{3}
			\item \pitch{へ/1,ん/0,な/0}(変な)
			\item \pitch{む/1,り/0,な/0}(無理な)
			\item \pitch{げ/1,ん/0,き/0,な/0}(元気な)
		\end{multicols}
		\begin{multicols}{3}
			\item \pitch{べ/1,ん/0,り/0,な/0}(便利な)
			\item \pitch{ぶ/1,べ/0,ん/0,な/0}(不便な)
			\item \pitch{し/1,ず/0,か/0,な/0}(静かな)			
		\end{multicols}
		\item \pitch{だ/1,い/0,す/0,き/0,な/0}(大好きな)
	\end{itemize}
	\item \textit{nakadaka}
	\begin{itemize}
		\begin{multicols}{2}
			\item \pitch{じ/0,ゆ/1,う/0,な/0}(自由な)
			\item \pitch{だ/0,い/1,じ\m{ょ}/1,う/0,ぶ/0,な/0,/0}(大丈夫な)
		\end{multicols}
	\end{itemize}
	\item \textit{odaka}
	\begin{itemize}
		\begin{multicols}{3}
			\item \pitch{す/0,き/1,な/0}(好きな)
			\item \pitch{い/0,や/1,な/0}(嫌な)
			\item \pitch{ら/0,く/1,な/0}(楽な)
			\item \pitch{じ\m{ょ}/0,う/1,ず/1,な/0}(上手な)
			\item \pitch{へ/0,た/1,な/0}(下手な)
			\item \pitch{だ/0,め/1,な/0}(だめな)
		\end{multicols}
	\end{itemize}
\end{enumerate}

Spajanje nadjeva sa さ nije toliko uobičajeno, već se umjesto koriste druge riječi (npr. umjesto \ruby{上手}{じょうず}さ koristi se \ruby{上手}{うま}さ) ili te riječi imaju nepravilne oblike (npr. \ruby{静}{しず}か --- \ruby{静}{しず}けさ).
Pravilo kod さ oblika nadjeva je da naglasak dolazi 1 m\=oru prije さ, osim tamo gdje ta m\=ora ne može nositi naglasak (tada 2 m\=ore prije さ), te u \textit{heiban} nadjevima, gdje naglasak ostaje nepromijenjen (\textit{heiban}).
Tako imamo:
\begin{itemize}
	\item \pitch{べ/1,ん/0,り/0,な/0}(便利な)、\pitch{べ/0,ん/1,り/1,さ/0}(便利さ)
	\item \pitch{じ/0,ゆ/1,う/0,な/0}(自由な)、\pitch{じ/0,ゆ/1,う/0,さ/0}(自由さ)
	\item \pitch{き/0,ら/1,い/1,な/1}(嫌いな)、\pitch{き/0,ら/1,い/1,さ/1}(嫌いさ)
\end{itemize}

\subsection{Naglasak kod osnovnih oblika idjeva}
\label{sub:idjevi}
Idjevi su, za razliku od nadjeva, pravi pridjevi, odnosno punokrvna vrsta riječi.
Razlikujemo ih od nadjeva po tom što imaju nastavak い koji se također mijenja ovisno o gramatičkom obliku, dok u slučaju nadjeva samo kopula ta koja se mijenja i ona nije sastavni dio riječi.

Idjevi eksponiraju vrlo osebujne uzorke naglasnih tipova, stoga je njihova temeljita kategorizacija dobra početna točka kod učenja njihovih naglasaka.

\subsubsection*{\textit{Atamadaka} idjevi}
U suvremenom japanskom jeziku ima jako malo \textit{atamadaka} idjeva.
Među onima koji se često susreću u svakodnevnom govoru su:
\begin{itemize}
	\begin{multicols}{4}
		\item \pitch{よ/1,い/0}、\pitch{い/1,い/0}(良い)
		\item \pitch{こ/1,い/0}(濃い)
		\item \pitch{お/1,お/0,い/0}(多い)
		\item \pitch{な/1,い/0}(無い)
	\end{multicols}
\end{itemize}
Oni su najbolji od svih idjeva zato što u izvedenim oblicima nikad ne mijenjaju naglasak.
Šteta što su samo četiri.

\subsubsection*{\textit{Nakadaka} idjevi}
\textit{Nakadaka} je najčešći naglasni tip među idjevima.
Važno je također da svi \textit{nakadaka} idjevi naglasak imaju točno na [-2] m\=ori, tj. jednoj prije završnog い.
Kako je ovo najbrojnija kategorija, vrijedi naučiti napamet one idjeve koje ne pripadaju ovoj kategoriji, a za ostale možemo pretpostaviti da su \textit{nakadaka}.
Primjeri:
\begin{itemize}
	\begin{multicols}{3}
		\item \pitch{あ/0,つ/1,い/0}(暑い、熱い)
		\item \pitch{か/0,わ/1,い/1,い/0}(かわいい)
		\item \pitch{お/0,も/1,し/1,ろ/1,い/0}(面白い)
	\end{multicols}
\end{itemize}

\subsubsection*{\textit{Odaka} idjevi}
\textit{Odaka} idjevi, koji bi naglasak imali na posljednjoj mori, い, ne postoje.
%	Oni bi možda bili još bolji od \textit{atamadaka} idjeva, ali šteta što ne postoje.

\subsubsection*{\textit{Heiban} idjevi}
Poseban žulj među ovim kategorijama predstavljaju, naravno, \textit{heiban} idjevi.
Neki od učestalih u svakodnevnom govoru su:
\begin{itemize}
	\begin{multicols}{3}
		\item \pitch{あ/0,か/1,い/1}(赤い)
		\item \pitch{あ/0,ま/1,い/1}(甘い)
		\item \pitch{お/0,も/1,い/1}(重い)
		\item \pitch{か/0,る/1,い/1}(軽い)
		\item \pitch{く/0,ら/1,い/1}(暗い)
		\item \pitch{あ/0,か/1,る/1,い/1}(明るい)
		\item \pitch{つ/0,め/1,た/1,い/1}(冷たい)
		\item \pitch{き/0,い/1,ろ/1,い/1}(黄色い)
		\item \pitch{む/0,ず/1,か/1,し/1,い/1}(難しい)
	\end{multicols}
\end{itemize}
No, \textit{heiban} idjevi se danas sve češće izgovaraju kao da su \textit{nakadaka} (s naglaskom na [-2] m\=ori).
Dakle, može i sljedeće:
\begin{itemize}
	\begin{multicols}{3}
		\item \pitch{あ/0,か/1,い/0}(赤い)
		\item \pitch{あ/0,ま/1,い/0}(甘い)
		\item \pitch{お/0,も/1,い/0}(重い)
		\item \pitch{か/0,る/1,い/0}(軽い)
		\item \pitch{く/0,ら/1,い/0}(暗い)
		\item \pitch{あ/0,か/1,る/1,い/0}(明るい)
		\item \pitch{つ/0,め/1,た/1,い/0}(冷たい)
		\item \pitch{き/0,い/1,ろ/1,い/0}(黄色い)
		\item \pitch{む/0,ず/1,か/1,し/1,い/0}(難しい)
	\end{multicols}
\end{itemize}

Međutim, prijevara leži u tome da se ova iznimka odnosi samo na osnovni oblik; svi izvedeni oblici ponašaju se prema pravilima specifičnim za \textit{heiban} naglasni tip.

\subsection{Naglasak kod izvedenih oblika idjeva}
Kako je prethodno spomenuto, idjevi završavaju nastavkom い koji se mijenja ovisno o gramatičkom obliku, te, za razliku od nadjeva, u idjevima se pritom može promijeniti i mjesto naglaska.
Evo glavnih izvedenih oblika i njihovih naglasaka s obzirom na prije navedene kategorije:

\subsubsection*{Izvedeni oblici \textit{atamadaka} idjeva}
Ranije je spomenuto da \textit{atamadaka} riječi ostaju \textit{atamadaka} i u izvedenim oblicima.
Evo i primjera:
\begin{itemize}
	\item \pitch{よ/1,い/0}、\pitch{よ/1,く/0}、\pitch{よ/1,さ/0}、\pitch{よ/1,く/0,て/0}、\pitch{よ/1,か/0,っ/0,た/0}、\pitch{よ/1,け/0,れ/0,ば/0}(良い)
	\item \pitch{こ/1,い/0}、\pitch{こ/1,く/0}、\pitch{こ/1,さ/0}、\pitch{こ/1,く/0,て/0}、\pitch{こ/1,か/0,っ/0,た/0}、\pitch{こ/1,け/0,れ/0,ば/0}(濃い)
	\item \pitch{お/1,お/0,い/0}、\pitch{お/1,お/0,く/0}、\pitch{お/1,お/0,さ/0}、\pitch{お/1,お/0,く/0,て/0}、\pitch{お/1,お/0,か/0,っ/0,た/0}、\pitch{お/1,お/0,け/0,れ/0,ば/0}(多い)
\end{itemize}

\subsubsection*{Izvedeni oblici \textit{nakadaka} idjeva}
\textit{Nakadaka} idjevi se također gotovo u potpunosti pravilno mijenjaju.
Postoji samo jedna iznimka, i to kod idjeva od 3 m\=ore: u izvedenim oblicima naglasak se s 2. m\=ore prebacuje na 1., odnosno oni postaju \textit{atamadaka}.
Primjer:
\begin{itemize}
	\item
	\pitch{あ/0,つ/1,い/0}: 
	\pitch{あ/1,つ/0,く/0}、
	\pitch{あ/1,つ/0,さ/0}、
	\pitch{あ/1,つ/0,く/0,て/0}、
	\pitch{あ/1,つ/0,か/0,っ/0,た/0}、
	\pitch{あ/1,つ/0,け/0,れ/0,ば/0}\\(暑い)
\end{itemize}

\textit{Nakadaka} idjevi s 4 ili više m\=ora zadržavaju naglasak na istom mjestu kao i u osnovnom obliku:
\begin{itemize}
	\item
	\pitch{か/0,わ/1,い/1,い/0}: 
	\pitch{か/0,わ/1,い/1,く/0}、
	\pitch{か/0,わ/1,い/1,さ/0}、
	\pitch{か/0,わ/1,い/1,く/0,て/0}、
	\pitch{か/0,わ/1,い/1,か/0,っ/0,た/0}、
	\pitch{か/0,わ/1,い/1,け/0,れ/0,ば/0}(可愛い)
	\item
	\pitch{お/0,も/1,し/1,ろ/1,い/0}: 
	\pitch{お/0,も/1,し/1,ろ/1,く/0}、
	\pitch{お/0,も/1,し/1,ろ/1,さ/0}、
	\pitch{お/0,も/1,し/1,ろ/1,く/0,て/0}、
	\pitch{お/0,も/1,し/1,ろ/1,か/0,っ/0,た/0}、
	\pitch{お/0,も/1,し/1,ろ/1,け/0,れ/0,ば/0}(面白い)
\end{itemize}

\subsubsection*{Izvedeni oblici \textit{heiban} idjeva}
Kako je prethodno spomenuto, \textit{heiban} idjevi u svom osnovnom obliku mogu se još izgovarati i kao \textit{nakadaka}.
Izvedeni oblici ponašaju se prema posebnim pravilima za \textit{heiban} idjeve.
Tako 〜く i 〜さ oblici ostaju \textit{heiban}, dok 〜くて, 〜かった i 〜ければ imaju naglasak na jednoj mori prije い u osnovnom obliku (kao da su \textit{nakadaka} idjevi).
Primjeri:
\begin{itemize}
	\item \pitch{あ/0,か/1,い/1}、\pitch{あ/0,か/1,い/0}(赤い): 
	\begin{itemize}
		\item \pitch{あ/0,か/1,く/1}、
		\pitch{あ/0,か/1,さ/1}, ali:
		\item \pitch{あ/0,か/1,く/0,て/0}、	
		\pitch{あ/0,か/1,か/0,っ/0,た/0}、
		\pitch{あ/0,か/1,け/0,れ/0,ば/0}
	\end{itemize}
	\item \pitch{か/0,る/1,い/1}、\pitch{か/0,る/1,い/0}(軽い): 
	\begin{itemize}
		\item \pitch{か/0,る/1,く/1}、
		\pitch{か/0,る/1,さ/1}, ali:
		\item \pitch{か/0,る/1,く/0,て/0}、
		\pitch{か/0,る/1,か/0,っ/0,た/0}、
		\pitch{か/0,る/1,け/0,れ/0,ば/0}
	\end{itemize}
\end{itemize}
